\subsection{08 января. г. Крутой Ключ}
\textit{Метеоусловия: весь день переменная облачность, без осадков}

\begin{figure}[h!]
	\centering
	\includegraphics[angle=90, width=0.7\linewidth]{pics/maps/08}
	\label{fig:08}
\end{figure}


Подъём дежурных в 07:00, общий подъём в 07:30, выход в 09:35. Начинаем движение по дороге вниз по течению р.~Сухокаменка, не забывая пропустить свой поворот. При повороте на восток пересекаем р.~Сухокаменка по снежному мосту, лыжи не снимаем, ноги не мочим (рис.~\ref{fig:20260108_095114.jpg}). Перед нами два пути: трек идёт по тропе южнее, которая ведет сразу к вершине, но несколько севернее есть дорога, которая ведет на седловую точку, откуда можно добраться до вершины. Выбираем второй вариант, так как по дороге несколько дней назад проезжал транспорт квадроцикл. Тропить, однако, приходится, так как уже несколько дней подряд идёт снег. ГТЛ = 15 см. Уклон очень приятный, не более 10\degree~ (\figurename~\ref{fig:20260108_100501.jpg}) За один переход (50 мин ЧХВ) выходим на седловину.

\begin{figure}[h!]
	\centering
	\includegraphics[width=0.7\linewidth]{pics/08/20260108_095114.jpg}
	\caption{Поворот от дороги на склон к в. Крутой Ключ}
	\label{fig:20260108_095114.jpg}
\end{figure}

\begin{figure}[h!]
	\centering
	\includegraphics[width=0.7\linewidth]{pics/08/20260108_100501.jpg}
	\caption{Движение по склону на седловую точку в. Крутой Ключ}
	\label{fig:20260108_100501.jpg}
\end{figure}

Через 50 м после седловинки тропа курумником уходит на хребет. Участок небольшой и неприятный: высота до 1 м с уклоном до 25\degree  (\figurename~\ref{fig:20260108_104601.jpg}). Идём пешком, подбиваем ступени, стараемся не провалиться между камнями.

\begin{figure}[h!]
	\centering
	\includegraphics[width=0.5\linewidth]{pics/08/20260108_104601.jpg}
	\caption{Подъём на хребет}
	\label{fig:20260108_104601.jpg}
\end{figure}

На гребне делаем привал, руководитель с участником налегке разведывают и протрапливают дорогу по хребту до полянки N 55.26504° E 60.00756°~--- начала подъёма на вершину. Далее движемся к полянке всей группой, любуемся видами на окружающие нас хребты (\figurename~\ref{fig:20260108_113611.jpg}, \ref{fig:img20260108113830}).

\begin{figure}[h!]
	\centering
	\includegraphics[width=0.5\linewidth]{pics/08/20260108_113611.jpg}
	\caption{Движение по хребту}
	\label{fig:20260108_113611.jpg}
\end{figure}

\begin{figure}[h!]
	\centering
	\includegraphics[width=0.9\linewidth]{pics/08/IMG20260108113830}
	\caption{Панорама хребта Большой Таганай с хребта в. Крутой Ключ}
	\label{fig:img20260108113830}
\end{figure}

В 11:50 начинаем подъём пешком налегке на вершину. Двигаемся справа пхд от огромной скалы, которую называют Зубом Дракона, далее наверх по курумнику. Выходим на вершину в~12:15.

\begin{figure}[h!]
	\centering
	\includegraphics[width=0.7\linewidth]{pics/08/kk}
	\caption{Путь подъёма и спуска}
	\label{fig:kk}
\end{figure}

\begin{figure}[h!]
	\centering
	\includegraphics[width=0.7\linewidth]{pics/08/photo_2026-01-09_21-53-54}
	\caption{Группа на в. Крутой Ключ}
	\label{fig:photo_2026-01-09_21-53-54}
\end{figure}

В 12:30 возвращаемся на полянку и устраиваем обед. В 13:00 начинаем спуск по треку. Тропы нет, но лес хорошо проходим. Первые 50 вертикальных метров идём пешком, затем встаём на лыжи.

\begin{figure}[h!]
	\centering
	\includegraphics[width=0.7\linewidth]{pics/08/20260108_133150}
	\caption{Спуск с хребта, идём пешком}
	\label{fig:20260108_133150.jpg}
\end{figure}

Несмотря на то, что мы находимся на треке, тропы не видно. Имеется лыжня, но она идёт не в наше сторону. Поэтому азимутом выходим к дороге (около 800 горизонтальных метров) и движемся по ней. Спустя СКОЛЬКО времени новая неприятность: на развилке N~55.25171\degree E~60.01929\degree не видим дороги, ведущей по нашему треку на запад на седловую точку, в наличии только дорога, идущая на юг вниз по течению ручья. Думаем и принимаем решение не тратить время на разведку, возможно, несуществующей дороги, а воспользоваться имеющейся дорогой и по ней выйти в долину сначала Куштумги, а затем и Сухокаменки, благо крюк при этом не превышает километра, а дороги обещают быть хорошими, так как по долине Куштумги идёт Южноуральская тропа и дорога в п. Северные Печи.

Наши ожидания, в целом, оправдались: по дороге вдоль ручья была несложная тропёжка с ГТЛ < 10 см (\figurename \ref{fig:20260108_152217.jpg}), а в самой долине тропёжка вскоре движением по колее квадроцикла (\figurename \ref{fig:20260108_155735.jpg}). Куштумгу бродить не пришлось, пересекаем её по снежному мосту.

\begin{figure}[h!]
	\centering
	\includegraphics[width=0.7\linewidth]{pics/08/20260108_152217}
	\caption{Спуск в д.р. Куштумга}
	\label{fig:20260108_152217.jpg}
\end{figure}

\begin{figure}[h!]
	\centering
	\includegraphics[width=0.7\linewidth]{pics/08/20260108_155735}
	\caption{Движение по д.р. Куштумга к слиянию с р. Сухокаменка}
	\label{fig:20260108_155735.jpg}
\end{figure}

Понимая, что до планового места ночёвки по светлому не дойдём, начинаем искать место для ночевки на слиянии р. Куштумга и Сухокаменка. 

Встаём на ночёвку в 17:05. Координаты м.н.: N 55.22560° E 59.98704°.

ОХВ: 07:30, ЧХВ: 

\clearpage