\subsection{08 января. г. Крутой Ключ}
\textit{Метеоусловия: }

\begin{figure}[h!]
	\centering
	\includegraphics[angle=90, width=0.7\linewidth]{pics/maps/08}
	\label{fig:08}
\end{figure}



Подъём дежурных в 07:00, общий подъём в 07:30, выход в 09:35. Начинаем движение по дороге вниз по течению р.~Сухокаменка, не забывая пропустить свой поворот. При повороте на восток пересекаем р.~Сухокаменка по снежному мосту, лыжи не снимаем, ноги не мочим (рис.~\ref{fig:20260108_095114.jpg}). Перед нами два пути: трек идёт по тропе южнее, которая ведет сразу к вершине, но несколько севернее есть дорога, которая ведет на седловую точку, откуда можно добраться до вершины. Выбираем второй вариант, так как по дороге несколько дней назад проезжал транспорт квадроцикл. Тропить, однако, приходится, так как уже несколько дней подряд идёт снег. ГТЛ = 15 см. Уклон очень приятный, не более 10\degree (\figurename~\ref{fig:20260108_100501.jpg}) За один переход \alert{(СКОЛЬКО)} выходим на седловину

\begin{figure}
	\centering
	\includegraphics[width=0.7\linewidth]{pics/08/20260108_095114.jpg}
	\caption{Поворот от дороги на склон к в. Крутой Ключ}
	\label{fig:20260108_095114.jpg}
\end{figure}

\begin{figure}
	\centering
	\includegraphics[width=0.7\linewidth]{pics/08/20260108_100501.jpg}
	\caption{Движение по склону на седловую точку в. Крутой Ключ}
	\label{fig:20260108_100501.jpg}
\end{figure}


\begin{figure}
	\centering
	\includegraphics[width=0.9\linewidth]{pics/08/IMG20260108113830}
	\caption{Панорама хребта Большой Таганай с хребта в. Крутой Ключ}
	\label{fig:img20260108113830}
\end{figure}



Встали на ночёвку в 17:05. Координаты м.н.: N 55.22560° E 59.98704°.

ОХВ: 07:30, ЧХВ: 

\clearpage