\section{Итоги похода, выводы и рекомендации по совершённому походу (от лица руководителя)}

\subsection{Выбор и прохождение маршрута}  
\begin{enumerate}
	\item Главное: по моему мнению, на Тескей-Ала-Тоо можно и нужно проводить горные <<единички>>. Этому способствует относительно простая логистика, разнообразие рельефа и опция пересечь главный хребет (в вашем распоряжении три недалеко отстоящих друг от друга перевалов 1А). Также <<изюминкой>> похода может быть обзорная точка района~--- вершина Марс. Однако стоит иметь в виду б\'{о}льшие, чем на Кавказе, высоты и более сложную, чем на Кавказе, логистику. Поэтому идеальным мне кажется вариант, в котором участники похода уже имеют горный или высотный опыт и хотя бы частично схожены, поскольку в ситуации, если что-то пойдёт не так, решать проблемы будет заметно сложнее, чем на Кавказе.
	\item Намеренно низкий темп движения (для этого ставили одним из первых самого неторопливого участника) в первые, в том числе перевальные, дни, по моему мнению, способствовал более хорошему общему самочувствию группы и более плавной акклиматизации. Оцениваю процесс акклиматизации в этом походе как более удачный и плавный по сравнению с прошлогодним походом по Гвандре \cite{Snegovskaya2024}. Общий тайминг дня также был удачным, время выхода было таким, что мы с нашим неспешным темпом вставали на бивак в среднем за несколько часов до заката и ни разу не ходили в темное время суток;
	\item Решение включить перевалы Саватор и Перемётный в нитку было не до конца моим, а возникло в силу необходимости частично <<развести>> нитку маршрута с Катиной. Эти перевалы явно сложнее среднестатистических 1А, и то, что они стояли первыми, не совсем хорошо. Тем не менее, для группы с уровнем физической и технической подготовки средний и выше, прохождение этих перевалов, в том числе и первыми, категорически рекомендую. Они разнообразны, интересны и малохожены, а на пер. Перемётный предоставляется замечательная возможность походить по пологому леднику в <<единичке>>; 
	\item Хоть ледник из предыдущего похода и пологий, несколько пар кошек на группу взять всё-таки ст\'{о}ит;
	\item Вершина Марс горячо рекомендуется к посещению. Несмотря на то, кто, кажется, виды с неё красивее во второй половине дня, рекомендую не рисковать и заходить на неё с утра;
	\item При планировании маршрута пятнадцатикилометровый забег по грунтовке от слияния р. Ашукашкасуу и Джууку до слияния р. Джууку и Джукучак казался мне скучным мероприятием для выполнения норматива. Но на месте я, и, как минимум, некоторые участники, были в восторге от красивых видов: красных скал, поросших густым зеленым кустарником. Здесь определённо стоит пройти и насладиться видами, чтобы глаз уставший от многодневних осыпей, отдохнул на более близкой к человеку природе.
	
\end{enumerate}


\subsection{Подготовка к походу и взаимодействие с группой} 
	\begin{enumerate}
		\item С технической стороны лимитирующими скорость группы факторами было неумение некоторых участников ходить по средней осыпи и осуществлять страховку альпенштоком~--- обычная история для новичковых походов. Поэтому следует по возможности и наличию места тренировки отрабатывать эти навыки заранее. Мы это делали весной в Приэльбрусье \cite{ostapiv2025}, и это подняло средний уровень альпенштокования в группе \smiley; 
		\item Уровень взаимодействия с группой на маршруте по вопросам стратегии и тактике  оцениваю, как минимум, как достаточный (или выше). Дело в том, что в прошлом походе по Гвандре \cite{Snegovskaya2024} недостаточность такого общения была для нескольних участников важной проблемой.
		
		 
		\item Так вышло, что с группой Кати шли раздельно (сначала~-- из-за нашего отставания на спуске с пер. Саватор, а после пер. Иттиш, фактически, намеренно, посколько наши группы уже стали достаточно самодостаточны, чтобы двигаться самостоятельно). Это скорее хорошо, чем плохо, так как управлять одной группой сложнее, да и внутри каждой из групп народ друг к другу <<притёрся>>, что привело к радикально разным уровням коммуникации в группах. Катина группа была значительно более активной и разговорчивой, в то время как мы были более спокойны и менее разговорчивы. Как следствие, объединение групп после пер. Иттиш было для нашей группы серьёзным испытанием, после чего мы всё-таки пошли раздельными группами;
		\item Возможно, в силу небольшого состава в группе поддерживался очень хороший психологический климат~--- лучший за все 4 предыдущих похода руководителя. Большая благодарность участникам за усилия, прилагаемые к этому \smiley
	\end{enumerate} 
	

	\clearpage