\section{Финансовый отчёт}

Раскладка была составлена на группу из 8-ми человек и 13 дней, включая два запасных. Средний вес 
продуктов в сухом виде на одного человека составил 536 гр./ 1943 ккал. в день, суммарный вес продовольствия на человека~--- 6.3 кг (без учёта заброски).
Заброска была спланирована на старте, выброска — вечером пятого дня. Участникам было рекомендовано брать 60 или более г карманки на день.
\\\\
Раскладка была составлена следующим образом:
\begin{itemize}
    \item Завтрак: гречка, булгур или овсянка по 45-60 г на человека либо омлет 30 г; сухмол, масло топлёное, сыр; один из 4-ёх видов орехов, такое же рахнообразие сухофруктов (и то и другое досыпалось в кашу) и чуть меньшее — сладкого.
    \item Обед: сухмясо 15-25 г на человека, суховощи, в качестве досыпки в суп шли кускус, рис пропаренный либо лапша; хлебцы, к ним различные виды колбас или сушёная рыба; сухофрукты.
    \item Ужин: пеммикан или сухмясо 35-45 г на человека; карпюр, киноа либо гречка 45-55 г, хлебцы, к ним сыр, курут либо два вида колбас, в один из дней сверх того — сало либо кедровые орехи (тем, кто сало не любит); одна из 4-ёх опций сладкого.
    \item Также брали приправы, перевальный шоколад и тушёнку "Кронидов" в качестве н/з.
    \item Калорийность по дням росла с 1737 ккал (499 г) на человека в первый день до 2066 ккал (593 г) в последний.
    \item Соотношение БЖУ колебалось около значения 20:20:60.
\end{itemize}
Сделанные выводы:
\begin{itemize}
    \item Раскладку можно было делать "голоднее", еды было много даже с учётом излишка некоторых видов продуктов из-за схода с маршрута трёх участников.
    \item Омлета нужно было закладывать 40-50 граммов вместо 30-ти, тогда это оправдало бы его постановку на завтрак наиболее тяжёлых дней как одновременно высококалорийный и быстроприготовляемый.
    \item На ужин, по просьбам медика, пили вопреки раскладке иванчай вместо чая зелёного/чёрного как наименее кофеинсодержащий напиток. В будущем 
имеет смысл сразу ставить на ужин различные успокаивающие травяные чаи.
    \item Какао и кисель пили не все.
    \item С огромным удовольствием участники ели карпюр с пеммиканом и жареным луком: в одном кане на пеммикановом жиру жарится лук, в другом — 
кипяток для разведения карпюра и чая. Изначальная нелюбовь начпрода к карпюру из-за его малой калорийности на единицу веса не выдержала испытания 
практикой.
    \item Кисель и какао идут хуже чаёв, ещё и каны после них отмывать.
    \item Киноа долго варится.
    \item Банан не очень хорош в качестве досыпки в кашу, т.к. пресный, без кислинки или тому подобной изюминки.
    \item Курут на любителя.
    \item В несколько особо тяжёлых дней на суповарение не было времени, и 
довольствовались сухой частью обеда. Поэтому имеет смысл закладывать сухой обед как минимум на запасные дни, чтобы в таких случаях заменять "мокрый"
обед полноценным сухим запасного дня.
    
\end{itemize}
