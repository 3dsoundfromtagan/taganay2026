\subsection{07 января. г. Ицыл}
\textit{Метеоусловия: }

\begin{figure}[h!]
	\centering
	\includegraphics[angle=0, width=0.7\linewidth]{pics/maps/07}
	\label{fig:07}
\end{figure}

Подъём дежурных в , общий подъём в , выход в 09:33. Переходим р.~Большой Киалим по деревянному мосту и движемся по маркированной тропе наверх. Средняя крутизна склона 12\degree, локально до 20\degree, на камусах проходится без проблем. ГТЛ = 10 см, хвост группы идёт по хорошо проторенной лыжне.

\begin{figure}[h!]
	\centering
	\includegraphics[width=0.7\linewidth]{pics/07/20260107_103632.jpg}
	\caption{Подъём по склону Ицыла}
	\label{fig:20260107_103632.jpg}
\end{figure}

Примерно за 100 горизонтальных и 20 вертикальных метров тропа теряется. Пытаемся тропить по приборам, но довольно быстро становится грустно это делать (ГТЛ более 30 см, уклон местами до 20\degree). Принимаем решение выйти траверсом на границу леса и курумника и по этой границе пешком подняться на хребет. Разведка занимает некоторое время, но, кажется, оно того стоит. Поднимаемся на хребет без проблем (\figref{fig:20260107_124239.jpg}), заодно прокладываем дорогу группе коммерческих туристов, поднимавшихся пешком от Киалимского кордона.

\begin{figure}[h!]
	\centering
	\includegraphics[width=0.7\linewidth]{pics/07/20260107_124239.jpg}
	\caption{Ползём по гребню Уральского хребта на вершину}
	\label{fig:20260107_124239.jpg}
\end{figure}

На хребте холодно, дует сильный ветер. Принимаем решение идти на северную вершину, так как с точки выхода на хребет до нее сильно ближе, и не нужно тропить (а только ползти по курумнику). Снимаем лыжи и выдвигаемся налегке. За локальным возвышением, отстоящим на 200 метров от точки выхода на хребет и на 100 метров~--- от северной вершины, решаем остановиться из-за сильного ветра (\figref{fig:20260107_124751.jpg}). Движение от точки выхода на хребет и обратно заняло у нас 15 мин ЧХВ.


\begin{figure}[h!]
	\centering
	\includegraphics[width=0.7\linewidth]{pics/07/20260107_124751.jpg}
	\caption{У вершины Ицыл Сев. }
	\label{fig:20260107_124751.jpg}
\end{figure}

Спуск начинаем в 13:30. Перед спуском выясняется, что у одного участника потерялась гайка лягушки (временно намертво приматываем ботинок к лыже скотчем), а у другого~--- порвался тросик (заменяет на свой запасной, лягушку не переставляет). Спускаемся на лыжах, благо склон некрутой, а лес редкий. Курс держим на юг, чтобы подсечь одну из троп, ведущих с хребта к избе Петр Севостьянова (N 55.32871° E 59.96697°). У избы есть страница ВК \cite{vk_sevastyanov}, где можно бесплатно <<забронировать>> её на необходимые даты, если хозяева избы сами в эти даты не планируют там находиться. Собственно, пересекая одну из троп, видим вместо ней раскатанный проспект, по которому отдыхающие из избы катаются с вершины. В избе обедаем, а её обитатели помогают нам подобрать нужную лягушачью гайку взамен потерянной. Обед и ремонт занимают примерно 50 минут, и в 15:00 продолжаем спуск с месту ночёвки. Дорога хорошо накатана снегоходами и квадроциклами, движение не вызывает проблем. У обзорной площадки с очаровательным названием <<Поганая поляна>> (N 55.32856° E 59.98209°) чуть ли не впервые  начала похода выглядывает солнце. Любуемся красивыми видами на восточные хребты.

\begin{figure}[h!]
	\centering
	\includegraphics[width=0.7\linewidth]{pics/07/20260107_152548}
	\caption{Не такая уж она и поганая (по крайней мере зимой)}
	\label{fig:20260107_152548.jpg}
\end{figure}

После поляны дорога спускается на борт будущей долины реки Сухокаменка до высоты 770 м и далее идёт траверсом. Дорога хорошая, и мы успеваем за 1.5 ч ЧХВ, к 17:00, пробежать 8 км и встать на запланированном месте ночёвки~--- на полянке у истока Сухокаменки. Удивило разве что, кажется, вдоль дороги течёт ручей, который местами выходит на поверхность и подтапливает дорогу. Пару раз такое место обходим по склону, но затруднений это не вызывает. За 500~м до сегодняшнего финиша у участника отрывается лыжное крепление, и этот участок он проходит пешком. Крепление было прикручено обратно вечером этого же дня.

\begin{figure}[h!]
	\centering
	\includegraphics[width=0.7\linewidth]{pics/07/20260107_165946.jpg}
	\caption{Состояние дороги в д.р.~Сухокаменка}
	\label{fig:20260107_165946.jpg}
\end{figure} 
 

Координаты м.н.: N 55.27246° E 59.98674°

ОХВ: ЧХВ: 10:14.

\clearpage