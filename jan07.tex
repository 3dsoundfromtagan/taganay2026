\subsection{07 января. г. Ицыл}
\textit{Метеоусловия: }

\begin{figure}[h!]
	\centering
	\includegraphics[angle=0, width=0.7\linewidth]{pics/maps/07}
	\label{fig:07}
\end{figure}

Подъём дежурных в , общий подъём в , выход в 09:33. Переходим р.~Большой Киалим по деревянному мосту и движемся по маркированной тропе наверх. Средняя крутизна склона 12\degree, локально до 20\degree, на камусах проходится без проблем. ГТЛ = 10 см, хвост группы идёт по хорошо проторенной лыжне.

\begin{figure}[h!]
	\centering
	\includegraphics[width=0.7\linewidth]{pics/07/20260107_103632.jpg}
	\caption{Подъём по склону Ицыла}
	\label{fig:20260107_103632.jpg}
\end{figure}

Примерно за 100 горизонтальных и 20 вертикальных метров тропа теряется. Пытаемся тропить по приборам, но довольно быстро становится грустно это делать (ГТЛ более 30 см, уклон местами до 20\degree). Принимаем решение выйти траверсом на границу леса и курумника и по этой границе пешком подняться на хребет. Разведка занимает некоторое время, но, кажется, оно того стоит. Поднимаемся на хребет без проблем (\figref{fig:20260107_124239.jpg}), заодно прокладываем дорогу группе коммерческих туристов, поднимавшихся пешком от Киалимского кордона.

\begin{figure}[h!]
	\centering
	\includegraphics[width=0.7\linewidth]{pics/07/20260107_124239.jpg}
	\caption{Ползём по гребню Уральского хребта на вершину}
	\label{fig:20260107_124239.jpg}
\end{figure}

\begin{figure}[h!]
	\centering
	\includegraphics[width=0.7\linewidth]{pics/07/20260107_124751.jpg}
	\caption{У вершины Ицыл Сев. }
	\label{fig:20260107_124751.jpg.jpg}
\end{figure}



Координаты м.н.: N 55.27246° E 59.98674°

ОХВ: ЧХВ 10:14.

\clearpage