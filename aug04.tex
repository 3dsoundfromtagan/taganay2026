\subsection{04 августа. д.р. Саватор}
\textit{Метеоусловия: утром, днём, вечером: переменная облачность, без осадков.}

\begin{figure}[h!]
	\centering
	\includegraphics[angle=0, width=0.45\linewidth]{pics/maps/04}
	\label{fig:04}
\end{figure}

Подъём в 05:00. Рассветает, можно ходить без фонарика, облачно, однако без осадков. Следы ночного дождя видны. Завершаем завтрак, сборы и выходим. Разделяемся с группой Кати. В 06:35 выходим из лагеря и начинаем подъём в д.р. Саватор. Местность лесистая, на подъёме отчётливо видна тропа. В некоторых местах тропа разветвляется, но трудности для ориентирования это не представляет.

\begin{figure}[h!]
	\centering
	\includegraphics[width=0.7\linewidth]{pics/04/IMG_2151}
	\caption{Подъём по лесной тропе в висячую долину}
	\label{fig:IMG_2151}
\end{figure}


Угол подъёма в среднем 20\degree, локально до 30\degree. На высотах 2776 и 2880 м есть отчётливые смотровые площадки.

\begin{figure}[h!]
	\centering
	\includegraphics[width=0.7\linewidth]{pics/04/IMG_2155}
	\caption{Вид на д.р. Чон-Кызыл-Суу со смотровой площадки}
	\label{fig:IMG_2155}
\end{figure}


В 08:10 выходим из зоны леса, высота 3000 м, здесь же заканчивапется ступень висячей долины и склон выполаживается. Движемся по правому берегу р. Саватор по правому берегу. Погода ясная, облачности нет. Провожаем группу Кати, к этому моменту уже переброжившую реку и поднимающуюся на склон для последующего переваливания в д.р. Каратакия.

\begin{figure}[h!]
	\centering
	\includegraphics[width=0.7\linewidth]{pics/04/IMG_2183}
	\caption{Движение в д.р. Саватор}
	\label{fig:IMG_2183}
\end{figure}

\begin{figure}[h!]
	\centering
	\includegraphics[width=0.7\linewidth]{pics/04/IMG_2218}
	\caption{д.р Саватор, вид на д.р. Чон-Кызыл-Суу}
	\label{fig:IMG_2218}
\end{figure}

Далее по пологой тропе без проблем движемся под моренную ступень на высоте 3450 м, где и встаём на ночёвку.

\begin{figure}[h!]
	\centering
	\includegraphics[width=0.7\linewidth]{pics/04/IMG_2226}
	\caption{д.р Саватор, вид на м.н.}
	\label{fig:IMG_2226}
\end{figure}


В 11:50 встаём лагерем месте ночёвки 2 (родина), перед моренной ступенью. Готовим обед, отдыхаем и собираемся в радиальный выход на моренную ступень для тренировки движения по крупной и средней осыпи. С верха моренной ступени сиден пер. Саватор.

\begin{figure}[h!]
	\centering
	\includegraphics[width=0.7\linewidth]{pics/04/IMG_2297}
	\caption{Верховья р. Саватор, вид с моренной ступени}
	\label{fig:IMG_2297}
\end{figure}

\begin{figure}[h!]
	\centering
	\includegraphics[width=0.7\linewidth]{pics/04/IMG_2273}
	\caption{Осыпные занятия: \textit{everybody gangsta until курумник}}
	\label{fig:IMG_2273}
\end{figure}

 Выход занимает 2 часа, с 14:00 по 16:00. За это время появилась облачность и незначительный дождь


ЧХВ: 3:45, ОХВ: 5:15
\clearpage