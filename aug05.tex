\subsection{05 августа. пер. Саватор (1А)}
\textit{Метеоусловия: утром ясно, днём туман, снежная крупа, вечером переменная облачность}

\begin{figure}[h!]
	\centering
	\includegraphics[angle=0, width=0.5\linewidth]{pics/maps/05}
	\label{fig:05}
\end{figure}

05:00~--- подъём. Завтракаем, собираем вещи и выходим в 07:10, погода ясная. Поднимаемся на моренную ступень по курумнику по маршруту, который выбрали в ходе вчерашнего радиального выхода.


\begin{figure}[h!]
	\centering
	\includegraphics[width=0.7\linewidth]{pics/05/IMG_2341}
	\caption{Подъём по курумнику}
	\label{fig:IMG_2341}
\end{figure}

8:00~--- Поднялись на моренную ступень (за 50 минут ЧХВ), открылся вид на перевал Саватор. Пока поднимались, ясное небо сменилось облаками, но пока без осадков.

8:10~--- Идём к взлёту на пер. Саватор. Стараемся обходить крупный курумник по зелёным участкам. Идти по ним несложно, уклон не более 15 градусов, но часто приходится переходить через осыпи крупного камня до следующих зелёных площадок.

\begin{figure}[h!]
	\centering
	\includegraphics[width=0.7\linewidth]{pics/05/IMG_2360}
	\caption{Движение ко взлёту на Саватор}
	\label{fig:IMG_2360}
\end{figure}

Чтобы обойти крупную каменистую осыпь, расположенную прямо под перевалом, поднялись на правый пхд борт цирка, траверсом вышли под перевальный взлёт и начали подъём.

\begin{figure}[h!]
	\centering
	\includegraphics[width=0.4\linewidth]{pics/05/IMG_2446}
	\caption{Начало движения по взлёту}
	\label{fig:IMG_2446}
\end{figure}

В течение 90 минут поднимаемся на Саватор по травянисто-осыпному склону. Уклон крутой 30--40\degree, движемся пологими траверсами с частыми остановками на отдых. На второй половине подъема \textit{погода резко испортилась}\textsuperscript{TM}, начался дождь со снегом с сильным ветром из долины, который несколько раз начинался и заканчивался, также выпал град.

\begin{figure}[h!]
	\centering
	\includegraphics[width=0.4\linewidth]{pics/05/IMG_2487}
	\caption{Движение по взлёту}
	\label{fig:IMG_2487}
\end{figure}


13:10~--- Поднялись на перевал. Перевальная седловина~--- разрушенное скальное ребро, сразу после подъёма открывается вид на спуск, также вдалеке был виден Иссык-Куль. Тур нашли на левой стороне перевальной седловины по ходу движения. На перевале пережидаем снова начавшийся мокрый снег (снежная крупа).
Мест для палаток нет.

\begin{figure}[h!]
	\centering
	\includegraphics[width=0.4\linewidth]{pics/05/IMG_2496}
	\caption{Седловина пер. Саватор}
	\label{fig:IMG_2496}
\end{figure}

\begin{figure}[h!]
	\centering
	\includegraphics[width=0.4\linewidth]{pics/05/IMG_2512}
	\caption{Группа на пер. Саватор,вид на д.р. Саватор}
	\label{fig:IMG_2512}
\end{figure}

\begin{figure}[h!]
	\centering
	\includegraphics[width=0.7\linewidth]{pics/05/IMG_2508}
	\caption{Группа на пер. Саватор,вид в сторону д.р. Киче-Кызыл-Суу (спуск)}
	\label{fig:IMG_2508}
\end{figure}	

13:40~--- Погода улучшается: облачно без осадков. Начинаем спуск по крутому склону с крупной и средней каменной осыпью, много подвижных камней, двигаемся плотной группой. Спуск занимает много времени, часто придерживаемся руками.

\begin{figure}[h!]
	\centering
	\includegraphics[width=0.7\linewidth]{pics/05/IMG_2532}
	\caption{Спуск с пер. Саватор}
	\label{fig:IMG_2532}
\end{figure}	

На второй половине спуска начинают встречаться зелёные участки. Стараемся двигаться по ним, но всё равно постоянно приходится выходить на курумник. Последний участок крутого спуска идём по сплошному курумнику среднего размера. Спуск по крутому участку занял 43 минуты чистого времени.

\begin{figure}[h!]
	\centering
	\includegraphics[width=0.8\linewidth]{pics/05/IMG_2594}
	\caption{Спуск с пер. Саватор}
	\label{fig:IMG_2594}
\end{figure}	

14:35~--- после крутого спуска продолжаем движению по пологому спуску (уклон меньше 10 градусов) по среднему курумнику (гребень моренного вала) в направлении травянистых террас. В определённый момент (N 42.13726° E 78.15254°) спускаемся с гребня в карман и далее до площадок движемся по нему.

17:10~--- Приходим на место ночёвки (высота 3580 метров). 

\begin{figure}[h!]
	\centering
	\includegraphics[width=0.7\linewidth]{pics/05/20250805_184953.jpg}
	\caption{м.н. 05--06.08}
	\label{fig:20250805_184953.jpg}
\end{figure}	




ОХВ: 10:00. ЧХВ: 7:45

Координаты м.н.: N 42.13999° E 78.14992°.

\clearpage