\subsection{08 августа. пер. Ашутор Западный}
\textit{Метеоусловия: }

\begin{figure}[h!]
	\centering
	\includegraphics[angle=0, width=0.7\linewidth]{pics/maps/08}
	\label{fig:08}
\end{figure}

\textit{Ашутор (кырг.)~--- <<Перевальное урочище>>}

Когда встали в 06:00, продолжался ночной дождь. После выхода на маршрут в 08:15 дождь закончился, но погода осталась облачной. Спускаемся к месту брода, бродим р. Джукучак и начинаем подъём по тропе на склон.

\begin{figure}[h!]
	\centering
	\includegraphics[width=0.8\linewidth]{pics/08/f20250808_083320}
	\caption{Брод р. Джукучак}
	\label{fig:f20250808_083320}
\end{figure}

09:20~--- Начали подъём по правому берегу ущелья, на дне которого течёт ручей. 

\begin{figure}[h!]
	\centering
	\includegraphics[width=0.8\linewidth]{pics/08/IMG_3104}
	\caption{Поднимаемся по травянистой тропе (уклон 10-15 градусов) по склону поросшему редкими деревьями }
	\label{fig:IMG_3104}
\end{figure}


\begin{figure}[h!]
	\centering
	\includegraphics[width=0.8\linewidth]{pics/08/IMG_3119}
	\caption{Ущелье при подъёме на Ашутор Западный}
	\label{fig:IMG_3119}
\end{figure}

 
В 09:46 начался мелкий дождь, тропа иногда теряется, движемся наверх траверсом. Дойдя до моста на правую пхд сторону каньона, переходим на него. На правой стороне продолжаем движение по узкой тропе по склону с короткой травой.


\begin{figure}[h!]
	\centering
	\includegraphics[width=0.8\linewidth]{pics/08/IMG_3131}
	\caption{Подъём по правому пхд борту ущелья}
	\label{fig:IMG_3131}
\end{figure}

11:52~--- подходим к перевальному взлёту, уклон становится круче (до 20 градусов).

\begin{figure}[h!]
	\centering
	\includegraphics[width=0.8\linewidth]{pics/08/IMG_3142}
	\caption{Подъём по правому пхд борту ущелья}
	\label{fig:IMG_3142}
\end{figure}

12:26~--- Поднялись на перевал Ашутор Западный. Перевальная седловина представляет собой неширокую, но плоскую площадку, удобную для привала. Воды нет.

\begin{figure}[h!]
	\centering
	\includegraphics[width=0.8\linewidth]{pics/08/IMG_3154}
	\caption{Группа на пер. Ашутор Западный, вид  д.р. Джукучак}
	\label{fig:IMG_3154}
\end{figure}

\begin{figure}[h!]
	\centering
	\includegraphics[width=0.8\linewidth]{pics/08/IMG_3157}
	\caption{Группа на пер. Ашутор Западный, вид  д.р. Ашукашкасуу}
	\label{fig:IMG_3157}
\end{figure}


В 12:49 начали спуск с перевала по границе красной и чёрной мелких осыпей. Тропа протоптана группой Кати. Двигаться несложно, тренируем спуск по <<лифтовой>> осыпи.

\begin{figure}[h!]
	\centering
	\includegraphics[width=0.6\linewidth]{pics/08/IMG_3164}
	\caption{Начало спуска. Двуцветная сыпуха}
	\label{fig:IMG_3164}
\end{figure}

Спустя 15 минут спуска мелкая осыпь заканчивается, тропа уходит в травянистый каньон пересохшего ручья, переходя с его левого берега на правый. Двигаться по нему несложно, в паре мест есть обходы водопадов высотой 1 м.

\begin{figure}[h!]
	\centering
	\includegraphics[width=0.7\linewidth]{pics/08/IMG_3167}
	\caption{Спуск. Каньон пересохшего ручья}
	\label{fig:IMG_3167}
\end{figure}

\begin{figure}[h!]
	\centering
	\includegraphics[width=0.7\linewidth]{pics/08/IMG_3186}
	\caption{Спуск. Каньон пересохшего ручья}
	\label{fig:IMG_3186}
\end{figure}

В 14:36 спустились на дно д.р. Ашуу–Кашка–Суу и остановились на обед. Начал накрапывать мелкий дождик. Нужно иметь в виду, что в реке вода не самая чистая, вокруг пасётся скот, а воду из ущелья взять можно, но она уходит под камни примерно на 150 м выше дна долины, поэтому о воде во время спуска по ущелью нужно позаботиться заранее.

В 16:05 вышли вниз по долине в направлении слияния р. Ашукашкасуу и Джууку. Тропа идёт по правому берегу реки, затем по мосту (координаты N 42.08843° E 77.99627°) переходим на левый берег. Тропа поднимается на борт долины в лес.

IMG 3230 мост через Ашуу–Кашка–Суу
Тропа местами узкая, но по большей части расширяется.


IMG 3244

17:08 – Переходим в брод реку Джаман-Суу. Несмотря на то что река выглядит глубокой и бурной, переход не вызвал проблем, вода была ниже колена.


IMG 3277 – Переход Джаман-Суу

После перехода реки выходим на лесную дорогу, по которой в 18:09 доходим до коша с заброской.
Не найдя места для стоянки рядом с кошем, перешли на другую сторону реки.


IMG 3288 – подходим к кошу по лесной колее.


\begin{table}[h!]
	\centering
	\begin{tabular}{|c|c|} 
		\hline 
		Этап & ЧХВ \\ 	
		\hline 
		Подъём по левой стороне каньона & 02:04 \\
		Подъём к взд=лёту перевала по правой стороне & 01:04 \\
		Подъём по перевальному взлёту & 00:24 \\
		Подъём по мелкой каменистой осыпи & 00:13 \\
		Подъём по травянистому склону до каньона и спуск по каньону & 01:18 \\
		\hline
		\textsc{Полное время подъёма на перевал} & 03:02 \\
		\textsc{Полное время спуска с перевала} & 01:31 \\
		\textsc{Полное время прохождения перевала} & 04:33 \\
		\hline
	\end{tabular}
	\caption{Расклад времени, пер. Ашутор Западный}
\end{table}

\textbf{Выводы и рекомендации:} пер. Ашутор Западный~--- классическая технически и физически несложная 1А. Рекомендуется к прохождению в обе стороны, в том числе как первый определяющий перевал на маршруте. В дождливую погоду стоит быть осторожным на травянистых склонах.

\clearpage