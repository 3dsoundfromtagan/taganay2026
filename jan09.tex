\subsection{09 января. оз. Тургояк, финиш}
\textit{Метеоусловия: переменная облачность, преимущественно ясно, солнце}


\begin{figure}[h!]
	\centering
	\includegraphics[angle=0, width=0.8\linewidth]{pics/maps/09}
	\label{fig:09}
\end{figure}

Подъём дежурных в 07:00, общий подъём в 08:00, выход в 09:15. Движемся на юг по дороге вдоль р.~Куштумга, ГТЛ менее 10 см (\figref{fig:20260109_105700.jpg}). Доходим до поворота на восток в д.р. Липовский ручей (N~55.19929° E~59.97865°) в 11:10 и  делаем привал, чтобы дроновод смотался к ур. Куштумга, чтобы там полетать (\figref{fig:kushtumga}).

\begin{figure}[h!]
	\centering
	\includegraphics[width=0.7\linewidth]{pics/09/20260109_105700.jpg}
	\caption{Состояние дороги в д.р.~Куштумга}
	\label{fig:20260109_105700.jpg}
\end{figure}


\begin{figure}[h!]
	\centering
	\includegraphics[width=0.7\linewidth]{pics/09/kushtumga}
	\caption{Вид на хребты со стороны урочища Куштумга}
	\label{fig:kushtumga}
\end{figure}

\begin{figure}[h!]
	\centering
	\includegraphics[width=0.6\linewidth]{pics/09/IMG20260109113341.jpg}
	\caption{Буранка на подходе к седловой точке}
	\label{fig:IMG20260109113341.jpg}
\end{figure}

Далее проходим пологую седловую точку и начинаем спуск по дороге в долине р.~Липовский ручей. Дороги иногда разделяются, но ведут в одну сторону. Часть пути с ветерком проходим по буранке, часть неглубоко тропим.

Спускаемся по долине к ЛЭП в 12:20, здесь появляется мобильная связь (N 55.19409° E 60.02239°). Далее движемся по буранке, а затем и по автомобильной дороге к б/о <<Серебряные пески>>, расположенной на побережье оз.~Тургояк.

\begin{figure}[h!]
	\centering
	\includegraphics[width=0.7\linewidth]{pics/09/IMG_4572}
	\caption{Группа выходит на лёд озера Тургояк}
	\label{fig:IMG_4572}
\end{figure}


\begin{figure}[h!]
	\centering
	\includegraphics[width=0.9\linewidth]{pics/09/20260109_152905.jpg}
	\caption{Народная тропа между о.~Веры и пос.~Тургояк}
	\label{fig:20260109_152905.jpg}
\end{figure}

\clearpage