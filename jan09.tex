\subsection{09 января. оз. Тургояк, финиш}
\textit{Метеоусловия: переменная облачность, преимущественно ясно, солнце}


\begin{figure}[h!]
	\centering
	\includegraphics[angle=0, width=0.7\linewidth]{pics/maps/09}
	\label{fig:09}
\end{figure}

Подъём дежурных в 07:00, общий подъём в 08:00, выход в 09:15. Принимаем решение построить маршрут сегодняшнего дня так, чтобы выйти к финишу не позже заката. Мотивация следующая: 
\begin{itemize}
	\item Вчера не дошли до места ночёвки 5 км, нагнать это расстояние по основному варианту маршрута за светлое время суток невозможно;
	\item Идти ночью по озеру не хотелось по соображениям безопасности
	\item Использовать запасной день не хотелось по причинам:
	\begin{itemize}
		\item Еженощного дождя из конденсата в палатке из-за её непромокаемых--недышаших стенок
		\item Печальной ситуации с сушинами в последние два дня
	\end{itemize}

\end{itemize}


Движемся на юг по дороге вдоль р.~Куштумга, ГТЛ менее 10 см (\figref{fig:20260109_105700.jpg}). Доходим до поворота на восток в д.р. Липовский ручей (N~55.19929° E~59.97865°) в 11:10 и  делаем привал, чтобы дроновод смотался к ур. Куштумга, чтобы там полетать (\figref{fig:kushtumga}).

\begin{figure}[h!]
	\centering
	\includegraphics[width=0.7\linewidth]{pics/09/20260109_105700.jpg}
	\caption{Состояние дороги в д.р.~Куштумга}
	\label{fig:20260109_105700.jpg}
\end{figure}


\begin{figure}[h!]
	\centering
	\includegraphics[width=0.7\linewidth]{pics/09/kushtumga}
	\caption{Вид на хребты со стороны урочища Куштумга}
	\label{fig:kushtumga}
\end{figure}

\begin{figure}[h!]
	\centering
	\includegraphics[width=0.6\linewidth]{pics/09/IMG20260109113341.jpg}
	\caption{Буранка на подходе к седловой точке}
	\label{fig:IMG20260109113341.jpg}
\end{figure}

Далее проходим пологую седловую точку и начинаем спуск по дороге в долине р.~Липовский ручей. Дороги иногда разделяются, но ведут в одну сторону. Часть пути с ветерком проходим по буранке, часть неглубоко тропим.

Спускаемся по долине к ЛЭП в 12:20, здесь появляется мобильная связь (N 55.19409° E 60.02239°). Далее движемся по буранке, а затем и по автомобильной дороге к б/о <<Серебряные пески>>, расположенной на побережье оз.~Тургояк.

На берегу озера обедаем, и в 13:20 выходим на лёд (\figref{fig:IMG_4572}). Курс держим на о.~Веры как на одну из походных достопримечательностей. В сторону острова ведёт буранка, идётся без проблем. По льду озера гуляют люди, катаются снегоходы, так что в прочности льда уверенность имеется.

\begin{figure}[h!]
	\centering
	\includegraphics[width=0.7\linewidth]{pics/09/IMG_4572}
	\caption{Группа выходит на лёд озера Тургояк}
	\label{fig:IMG_4572}
\end{figure}


До о.~Веры доходим за 30 мин ЧХВ. На острове руководитель, как местный житель, устраиваем нам экскурсию. Гуляем, смотрим на мегалиты (а некоторые и не только \smiley\figref{fig:megalit}) и остатки старообрядческого скита.

\begin{figure}[h]
	\begin{minipage}[t]{0.49\linewidth}
		\centering
		\includegraphics[width=\linewidth, height=8cm, keepaspectratio]{pics/09/IMG_4591}
		\caption{Спелеологи в естественной среде обитания}
	\end{minipage}
	\hfill
	\begin{minipage}[t]{0.49\linewidth}
		\centering
		\includegraphics[width=\linewidth, height=8cm, keepaspectratio]{pics/09/photo_2026-01-22_22-10-22.jpg}
		\caption{Миниатюра <<Группа и мегалиты>>}
	\end{minipage}
	\label{fig:megalit}
\end{figure}



\begin{figure}[h!]
	\centering
	\includegraphics[width=0.9\linewidth]{pics/09/ozero}
	\caption{Народная тропа между о.~Веры и пос.~Тургояк}
	\label{fig:ozero}
\end{figure}

Выдвигаемся от острова в сторону посёлка в 15:15. Дорога прекрасная, гуляют отдыхающие, от спортсменов, бегущих коньком на лыжах, до бабушек с в шубах и с клатчиками. Из-за большого количества людей чувствуем себя немного не в своей тарелке.

В 16:15 выходим на пляж с.~Тургояк, где нас встречают Сашины родственники и отвозят в Миасс (\figref{fig:photo_2026-01-22_22-28-36.jpg}). Поход завершён!

\begin{figure}[h!]
	\centering
	\includegraphics[width=0.7\linewidth]{pics/09/photo_2026-01-22_22-28-36.jpg}
	\caption{На финише}
	\label{fig:photo_2026-01-22_22-28-36.jpg}
\end{figure}

\clearpage