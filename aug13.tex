\subsection{13 августа. Финиш}
\textit{Метеоусловия: утром, ясно, тепло, днём сильный, затем затяжной дождь, вечером переменная облачность}


\begin{figure}[h!]
	\centering
	\includegraphics[angle=270, width=0.5\linewidth]{pics/maps/13}
	\label{fig:13}
\end{figure}

Проснулись на высоте 3310 м в 6.00. Погода ясная. Позавтракали, собрались, вышли на маршрут в 8.25. Грустно было покидать это уютное место ночёвки на маленькой лужайке между двумя ручьями, с которой открывался прекрасный вид на скалы и ледники. Скот сюда почти не добирался, так что трава осталась высокой, цветущее разнотравье радовало глаз.  

Начали спуск вдоль ручья по едва заметной тропинке, оставленной предыдущей группой. В 9.08 остановились на привал (ходовое время 43 мин) на высоте 3165 м.


\begin{figure}[h!]
	\centering
	\includegraphics[width=0.9\linewidth]{pics/13/20250813_090014.jpg}
	\caption{Начало спуска по древней морене}
	\label{fig:20250813_090014.jpg}
\end{figure}


В 09.19 продолжили спуск. Тропа часто теряется, но склон пересечён коровьими тропами, по которым часто идти было легче, чем просто вниз по склону. В 09.45 миновали последнюю древнюю морену и остановились на привал (ходовое время 47 мин) на высоте 2977 м. 

\begin{figure}[h!]
	\centering
	\includegraphics[width=0.6\linewidth]{pics/13/IMG_3646.jpg}
	\caption{Конец спуска по древним моренам}
	\label{fig:IMG_3646.jpg}
\end{figure}


В 10.48 прошли поворот с Ашутора Западного и остановились на привал (ходовое время 24 мин) на высоте 2885 м. Дальнейший маршрут до коша мы уже проходили 08.08.2025 после спуска с вышеупомянутого перевала. Долина реки здесь широкая, пасутся коровы и лошади, так что трава представляет собой коротко подстриженный коврик из манжетки и злаков. Набитая тропа ведет вниз по долине примерно в 50 м от русла реки. 


\begin{figure}[h!]
	\centering
	\includegraphics[width=0.7\linewidth]{pics/13/IMG_3649.jpg}
	\caption{Движение по д.р. Джукучак}
	\label{fig:IMG_3649.jpg}
\end{figure}

В 11.06 продолжили спуск по долине реки. В 12.30 спустились к кошу, где забирали заброску в конце первого кольца, перешли реку по мосту, остановились на обед на поляне около большого камня (ходовое время 1 час 24 минуты, высота 2535 м). Соединились с группой Кати. Переждали грозу и пообедали. 

До финиша оставался 15-километровый забег по хорошей грунтовке.
В 14.00 вышли вниз по долине одной большой группой по проезжей для легковых машин дороге, шли под слабым дождём. Нам встречаются группы российских и западных туристов, которых забрасывают сюда на джипах. В 14.55 сделали привал на высоте 2405 м на обочине дороги (ходовое время 55 мин). 

\begin{figure}[h!]
	\centering
	\includegraphics[width=0.8\linewidth]{pics/13/20250813_150637.jpg}
	\caption{д.р. Джууку}
	\label{fig:20250813_150637.jpg}
\end{figure}

В 15.00 продолжили спуск до высоты 2280 м. В 15.40 сделали привал (ходовое время 40 мин). Дождь прекратился, переменная облачность. В 16.06 продолжили спуск до высоты 2160 м. В 16.53 сделали привал (ходовое время 47 мин).

\begin{figure}[h!]
	\centering
	\includegraphics[width=0.9\linewidth]{pics/13/20250813_160451.jpg}
	\caption{д.р. Джууку}
	\label{fig:20250813_160451.jpg}
\end{figure}

\begin{figure}[h!]
	\centering
	\includegraphics[width=0.9\linewidth]{pics/13/20250813_173224.jpg}
	\caption{д.р. Джууку}
	\label{fig:20250813_173224.jpg}
\end{figure}


В 17.03 отправились дальше вниз до высоты 2000 м. В 18.10 пришли на финиш (ходовое время 1 час 7 мин). Сложили вещи на берегу реки напротив красных скал. Приготовили ужин в ожидании трансфера.

\begin{figure}[h!]
	\centering
	\includegraphics[width=0.9\linewidth]{pics/13/finish.jpg}
	\caption{д.р. Джууку}
	\label{fig:finish.jpg}
\end{figure}

В 20.00 погрузились в автобус. В 21.15 приехали в посёлок Тосор на высоте 1600 м (ходовое время 1 час 15 мин). Поселились в гостевом доме со стандартными номерами (санузел в номере). Наконец-то помылись. На этом активная часть похода завершена, нас ждёт отдых на Иссык-Куле!

ЧХВ: 06:47, ОХВ: 09:45

\clearpage 
