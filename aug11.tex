\subsection{11 августа. Сырты}
\textit{Метеоусловия: утром, днём, вечером ясно, тепло. В середине дня жарко. Вечером на месте стоянки дует сильный ветер.}


\begin{figure}[h!]
	\centering
	\includegraphics[angle=0, width=0.5\linewidth]{pics/maps/11}
	\label{fig:11}
\end{figure}

Подъём дежурных в 05:00, общий подъём в 05:30. Выходим на маршрут в 06:55. С 07:00 до 07:10 преодолеваем вброд полноводный ручей (N 42.00227\degree~ E 77.99665\degree), затем движимся по д.р. Ит-тиши. Движение по долине периодически осложняется необходимостью пересекать заболоченные участки. Постепенно удаляемся дальше от реки и забираем к борту долины, образованному отрогом хребта. В 09:45 выходим к небольшому ручью и устраиваем около него привал-перекус на 20 минут (N 41.96478\degree~ E 78.04114\degree). 

\begin{figure}[h!]
	\centering
	\includegraphics[width=0.69\linewidth]{pics/11/IMG_3508.jpg}
	\caption{Д.р. Ит-тиши}
	\label{fig:IMG_3508.jpg}
\end{figure}

К 11:10 после плавного подъёма выходим на безымянный н/к перевал (4035 м), разделяющий д.р. Ит-тиши и озёра Кашкасу. Пройдя некоторое время по плато, начинаем спуск по тр.-ос. склону к первому из озёр, у которого в 12:20 устраиваем привал на обед. 

\begin{figure}[h!]
	\centering
	\includegraphics[width=0.7\linewidth]{pics/11/IMG_3526.jpg}
	\caption{Вид на хр. Акшийрак с седловины н/к пер.}
	\label{fig:IMG_3526.jpg}
\end{figure}

\begin{figure}[h!]
	\centering
	\includegraphics[width=0.7\linewidth]{pics/11/IMG_3522.jpg}
	\caption{Вид на в. Марс и Джукучак с седловины н/к пер.}
	\label{fig:IMG_3522.jpg}
\end{figure}

\clearpage

\begin{figure}[h!]
	\centering
	\includegraphics[width=0.7\linewidth]{pics/11/IMG_3539.jpg}
	\caption{Спуск с н/к пер. к месту обеда у озера Кашкасу}
	\label{fig:IMG_3539.jpg}
\end{figure}

В 14:27 выдвигаемся в дорогу. Идём вдоль озёр Кашкасу. Местами движение затруднено заболоченностью берега. По возможности обходим заболоченные участки. К 16:10 доходим до ровного сухого участка недалеко от точки на карте мн8, на котором есть оборудованные места под пару палаток. Ставим лагерь, отдыхаем и ужинаем.

Координаты м.н.: N 42.00425\degree~ E 78.07789\degree.  

ЧХВ: 5:46, ОХВ: 9:15.

\begin{figure}[h!]
	\centering
	\includegraphics[width=0.7\linewidth]{pics/11/IMG_3556.jpg}
	\caption{Вид на район пер. Кашкасу с места ночёвки 11-12.08}
	\label{fig:IMG_3556.jpg}
\end{figure}

\begin{figure}[h!]
	\centering
	\includegraphics[width=0.7\linewidth]{pics/11/IMG_3557.jpg}
	\caption{Вид с м.н. на подход к началу подъёма на в. Марс}
	\label{fig:IMG_3557.jpg}
\end{figure}

%морене и выходим на горизонтальную осыпь на высоте $\sim3700$~м. Оставшуюся часть дня двигаемся по этой осыпи.

\clearpage