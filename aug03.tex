\subsection{03 августа. Старт}
\textit{Метеоусловия: утром, днём солнечно, вечером дождь, град}

\begin{figure}[h!]
	\centering
	\includegraphics[angle=0, width=0.4\linewidth]{pics/maps/03}
	\label{fig:03}
\end{figure}

В течение дня движение с группой Кати Тюриной совместное.

Подъём в 09:00. В 50~м от места ночёвки находится кур. Джилысу с тёплыми сероводородными ваннами. Место, похоже, популярное~--- с утра мимо лагеря проходит и проезжает много людей. Завершаем завтрак, сборы и перераспределение снаряжения. 

В 11:36 выходим из лагеря по грунтовой дороге на правой стороне реки Чон-Кызыл-Суу. Грунтовка твёрдая, широкая, активно используемая, с небольшими перепадами высоты. 

В 11:55 бродим левый приток р. Чон-Кызыл-Суу. Горный ручей образует на дороге широкий разлив (глубиной по щиколотку), 5 метрами выше есть возможность пересечь его по бревну. 

\begin{figure}[h!]
	\centering
	\includegraphics[width=0.7\linewidth]{pics/03/IMG_2049}
	\caption{Место брода}
	\label{fig:IMG_2049}
\end{figure}





В 12:05 возобновляем движение. В 12:10 переходим по мосту на правую сторону реки. По ходу маршрута встречается ещё несколько ручьёв, но они без проблем переходятся по камням.

\begin{figure}[h!]
	\centering
	\includegraphics[width=0.7\linewidth]{pics/03/IMG_2107}
	\caption{Дорога в д.р. Чон-Кызыл-Суу}
	\label{fig:IMG_2107}
\end{figure}


В 13:20 переходим по мосту на левую сторону реки Чон-Кызыл-Суу. Выходим в долину на пастбища, нас встречают коровы, лошади и ослы. Здесь же находится ФГС (Физико-географичесская станция).

\begin{figure}[h!]
	\centering
	\includegraphics[width=0.7\linewidth]{pics/03/IMG_2118}
	\caption{ФГС, вид вниз по течению}
	\label{fig:IMG_2118}
\end{figure}


\begin{figure}[h!]
	\centering
	\includegraphics[width=0.7\linewidth]{pics/03/IMG_2124}
	\caption{Разлив р. Чон-Кызыл-Суу сразу за ФГС}
	\label{fig:IMG_2124}
\end{figure}



В 14:00 проходим мимо коша возле слияния р. Каратакия и р. Чон-Кызыл-Суу. В 14:37 приваливаемся возле слияния р. Саватор и р. Чон-Кызыл-Суу для поиска места ночёвки. Хочется найти место повыше, где коровы уже не пасутся возле воды. 

В 14:45 выдвигаемся к месту ночёвки. Пересекаем  р. Саватор: хоть и много камней, но ботинки не замочить сложно.

В 15:05 обедаем на месте ночёвки. Место просторное, защищённое от ветра деревьями, но с заметным уклоном. Получились комфортно разместить 4 палатки. В северо-западном конце долины показались тучи.
Часть группы вернулась к кошу приобрести свежее молоко. Обменный курс: 3 сигареты + 1 баунти + беседа на 10~л парного молока. В ассортименте также есть сметана.

В 16:47 часть группы вышла на радиальный подъём по пути завтрашнего подъёма в д.р. Саватор. Гроза дошла до места ночёвки, сильный дождь, град. В 18:00 радиальщики поднялись на 300~м до отметки 2930~м, до границы зоны леса, и начали спуск. В 19:10 вернулись к месту ночёвки. Дождь прекратился и сменился переменной облачностью.

Ужин и после 20:00 отбой. Ночью дождь несильно лил ещё несколько раз.
 
 
\begin{figure}[h!]
	\centering
	\includegraphics[width=0.7\linewidth]{pics/03/DJI_0029}
	\caption{м.н. 3-4.08. Стрелкой показан подъём в д.р. Саватор}
	\label{fig:DJI_0029}
\end{figure}
 
ЧХВ 3:30, ОХВ 2:44. Координаты м.н.: N 42.17086° E 78.19931°.

\clearpage