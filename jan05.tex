\subsection{05 января. в. Круглица}
\textit{Метеоусловия: утром пасмурно, слабый снег; днём ясно, вечером переменная облачность, ночью снег}

\begin{figure}[h!]
	\centering
	\includegraphics[angle=0, width=0.6\linewidth]{pics/maps/05}
	\label{fig:05}
\end{figure}

Подъём дежурных в 05:30, общий подъём в 06:00, выход в 08:05. Движемся от приюта по Верхней Таганайской тропе на северо-восток вдоль хребта. Встречаем развилку троп с курумником и без него, выбираем второй вариант, хоть он и немного длиннее. К моменту, когда тропа поворачивает на склон к приюту Гремячий Ключ (N 55.27314° E 59.79999°) светает окончательно. Сам приют проходим в СКОЛЬКО. Отдыхающие гуляют до Откликного гребня, поэтому до него ведет очень накатанная и нахоженная дорога. В 10:50 проходим мимо подножия этой горы (\figref{fig:IMG20260105105051.jpg})

\begin{figure}[h!]
	\centering
	\includegraphics[width=0.7\linewidth]{pics/05/IMG20260105105051.jpg}
	\caption{г.~Откликной гребень. Стрелкой указано направление движения группы}
	\label{fig:IMG20260105105051.jpg}
\end{figure} 


Далее по хребту дорога менее укатанная, буранка заканчивается, но идти несложно. Движемся к перевалу Долина Сказок. По пути встречается два достаточно крутых, но коротких спуска, которые проходим без лыж. На перевал приходим в 12:30 (\figref{dolskazok}), обедаем. Решаем сходить на г.~Круглица радиально и спуститься с перевала к приюту <<Таганай>> и двигаться дальше по Нижней Таганайской тропе. Такое решение было принято в силу того, что при сквозном прохождении вершины и спуске к приюту <<стоянка Гарбера>> мы бы не уложились в светлое время суток и, скорее всего, сильно бы устали.

\begin{figure}[h!]
	\centering
	\includegraphics[width=0.7\linewidth]{pics/05/dolskazok.jpg}
	\caption{Группа на пер.~Долина Сказок}
	\label{fig:dolskazok.jpg}
\end{figure} 

\begin{figure}[h!]
	\centering
	\includegraphics[width=0.8\linewidth]{pics/05/IMG20260105140043_01.jpg}
	\caption{Группа на г.~Круглица}
	\label{fig:IMG20260105140043_01.jpg}
\end{figure} 


ОХВ:  ЧХВ: 08:00

Координаты м.н.: N 55.30319° E 59.86285°

\clearpage