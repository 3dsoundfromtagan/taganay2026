\section{Организация и проведение похода}
\subsection{Цели и задачи маршрута. Выбор нитки маршрута}
При разработке и планировании маршрута я, как руководитель, руководствовался следующими соображениями:
\begin{enumerate} 
	\item \textbf{Высотность, транспортная доступность горного района и стоимость трансфера.}
	Подавляющая часть группы (шесть человек из восьми) имела официальный или неофициальный опыт горных походов до 2 к.с. по Кавказу \cite{Snegovskaya2024} или Алтаю, и вся группа имела высотный опыт ночёвки не менее 2900 м \cite{ostapiv2025}. Это позволило выбрать в качестве района новичкового похода не Гвандру или Архыз, как это обычно бывает, а новый для участников и руководителя горный район Центральной Азии~--- Тескей-Ала-Тоо. Высоты в этом районе, в среднем, на 500 м больше, чем на Кавказе, а его транспортную доступность можно назвать приемлемой для новичковых походов: 4 часа самолётом до Бишкека и еще 7~--- до места старта в горах. Стоимость трансфера выше, чем на Кавказе, но ниже, чем на Алтае. 
		
	\item \textbf{Концентрированность препятствий.}
	Дополнительным фактором в сторону выбора Тескей-Ала-Тоо послужило также и то, что в отличие, например, от Алтая, хребтовка района (основной хребет и серия параллельных отрогов), как и на Кавказе, позволяет осуществить подход под многие перевалы в течение одного дня, миновав долгие и занудные забеги по долинам, и, как следствие, поддерживать интерес группы на приемлемом уровне.
	
	\item \textbf{Разнообразие рельефа.} 
	С методической точки зрения, а также, опять-таки, для поддержания интереса группы, хотелось продемонстрировать участникам как можно больше разнообразных типов рельефа. Район позволяет в походах 1 к.с. продемонстрировать все виды рельефа, за исключением, пожалуй, снежного. <<Изюминкой>> района является возможность пересечения главного хребта через перевалы 1А (разительно отличающихся от классических <<единичек А>>) с осмотром высокогорных болот~--- сыртов.
	
	\item \textbf{Эффект кульминации.}
	У меня как у руководителя было глубокое убеждение, что первый поход должен обладать понятным, с позволения сказать, сюжетом и иметь свою кульминацию. В нашем случае таким сюжетом было движение через отроги главного хребта с преодолением разнообразных перевалов (последовательно: травянисто-осыпной, ледовый, травянистый), а затем~--- кольцо через главный хребет с <<неклассическими>> перевалами на южную сторону, восхождением на обзорную точку~--- вершину Марс. Конец маршрута~--- забег по среднегорью с финишем на локальной достопримечательности <<Красные скалы>>~--- также должен был добавить в поход разнообразия и красоты. Наконец, по моему глубокому убеждению, финальным аккордом походов по Кыргызстану должен быть хотя бы  однодневный (а лучше двухдневный) отдых на Иссык-Куле.
	
	\item\textbf{Помощь в планировании.}
	Фактор, который формально не был в списке определяющих критериев, но по факту являлся таковым~--- это искренняя заинтересованность и помощь в планировании и организации маршрута от человека, через которого организовывалась логистика похода~--- Юрия Траченко. За эту помощь выражаю ему огромную благодарность.
	
\end{enumerate} 

\subsection{Аварийные выходы из маршрута и его запасные варианты}
\textbf{Аварийными выходами} с маршрута являлись:
\begin{itemize}
	\item На месте старта: спуск по д.р. Чон-Кызылсу;
	\item После первого пер.: спуск по д.р. Кичи-Кызылсу;
	\item После второго пер.: спуск по д.р. Джуукучак;
	\item После третьего пер.: спуск по д.р. Ашукашкасу и Джууку;
	\item На южной стороне хребта: спуск к КПП комбината <<Кумтор>>.
\end{itemize}
\textbf{Запасными вариантами} маршрута являлись:
\begin{itemize}
	\item Замена пер. Саватор (1А, 3860) на пер. \textbf{Каратакия (1А, 3800)};
	\item Замена пер. Перемётный (1А, 4026) на пер. \textbf{Кашкатор Северный (1А, 3929)};
	\item Отказ от рад. восхождения на верш. \textbf{Марс (н/к, 4334)}.
\end{itemize}


\subsection{Обеспечение безопасности на маршруте}
Группа была зарегистрирована в отделении МЧС России \alert{каком?}.

Связь группы с внешним миром (с МЧС и координатором) осуществлялась по мобильной связи. Она присутствует на старте (Центральная Усадьба Таганай), на финише (от ЛЭП в 1 км от северного побережья оз. Тургояк и далее), а также на всех вершинах и хребтах (Перья, Круглица, Дальний Таганай, Ицыл, Крутой Ключ).

Связь внутри группы (координация авангарда и арьергарда) осуществлялась при помощи раций.

\subsection{Логистика}

\textbf{Заезд на маршрут:}
Основная часть группы добиралась до г. Златоуст поездом 014 со временем прибытия в город 04.01 04:12, затем на автомобилях к Центральной усадьбе Таганая. Один участник добирался самолётом до Челябинска (время прилёта с учетом задержки на час~--- 06:00), где его забрали родственники руководителя, и вместе с руководителем и ещё одним участником на личных автомобилях доставили к Центральной усадьбе Таганая (общее время в пути от аэропорта~--- около 2.5~ч). По поводу перелёта следует иметь в виду, что у <<Аэрофлота>> есть опция бесплатного провоза горных лыж в чехле. Содержимое чехла не проверяется, тем самым можно частично разгрузить основной багаж, если в этом есть потребность.

\textbf{Выезд с маршрута: }
Логистика в цивилизацию была полностью организована с помощью родственников руководителя. От финиша (пляж в пос. Тургояк) до коттеджа в Миассе группу отвезли на трёх автомобилях, а рюкзаки и лыжи~--- на грузовом автомобиле. Время в пути~--- около 20 минут. Из Миасса в Москву основная часть группы добиралась поездом 013 со временем отбытия из Миасса 11.01. 00:05, прибытия в столицу 12.01 06:10.


\textbf{Проживание в Миассе:}
Группа сняла коттедж в Миассе (ул. Мирная, 19) на одну ночь (09--10.01) с опцией позднего выезда. Стоимость~---~20000~\faRuble.


\textbf{Связь:}
Приобретены местные SIM-карты оператора О!, использовался спутниковый телефон Iridium и рация для связи с параллельно идущей группой Тюриной Кати.

\subsection{Перечень наиболее интересных природных и исторических объектов, занятий на маршруте}
\begin{enumerate}[noitemsep,topsep=0pt,parsep=0pt,partopsep=0pt]
	\item гора \textbf{Перья} носит своё название за сходство с крылом летящей птицы.
	\item гора \textbf{Откликной Гребень} названа так за способность отражать семикратное эхо.
	Гребень выделяется в системе гор как спина гигантского окаменевшего
	дракона, с высотой скал до 150 метров.
	\item гора \textbf{Круглица}~---высшая и центральная точка Большого
	Таганая, тупоконическая вершина которой сложена массивными каменными
	глыбами, вес которых может достигать несколько десятков тонн. Северное
	плечо горы представляет собой почти идеально плоскую площадку на высоте
	около 1100~м., затянутую горной тундрой с произрастанием голубики ряда
	эндемичных и реликтовых видов.
	\item гора \textbf{Дальний Таганай}~--- при подъёме на неё пересекаются четыре климатические зоны: от смешанных лесов у подножия до тундры на вершине. Тундровый ландшафт окружают причудливые скальные
	останцы, переходящие через распадок в километровый массивный гребень
	(местное название «Волчий»), уходящий своими южными отрогами в долину
	реки Малый Киалим.
	\item гора \textbf{Ицыл} «вечный ветер». Интересна своими реликтовыми ельниками восточного склона горы. Является частью Уральского хребта, по горе проходит, по одной из классификаций, граница Европы и Азии.
	\item гора \textbf{Крутой Ключ} расположена на Южной оконечности хребта Малый Урал, но выделяется на фоне окружающих хребтов тем, что в отличие
	от них имеет ярко выраженную скальную вершину, а не покрыта лесом. Поэтому с нее открывается великолепный вид на весь горный массив
	Таганай, Ильменский хребет, долину реки Миасс, озеро Тургояк. Пик горы представляет собой скалу высотой 25 метров, протяженностью 50--70 метров.
	Вершина относительно доступна, до ее подножия можно доехать на внедорожнике или снегоходе с поселка Северные печи.
	\item остров \textbf{Веры} интересен своими древними мегалитическими сооружениями и руинами старообрядческих скитов;
	\item озеро \textbf{Тургояк}. Высота над уровнем моря 320~м, длина 6.9~км,
	ширина 6.3~км, площадь 26.4~км$^2$, длина береговой линии 27~км,
	наибольшая глубина 34~м, средняя глубина 19.2~м. Переход по льду озера к финишу~--- посёлку Тургояк~--- эффектная точка в завершение похода.
\end{enumerate}

\paragraph{Темы практических занятий:}

\begin{itemize}
	\item Техника передвижения на лыжах по горизонтальной местности, техника подъёмов и спусков
\end{itemize}

\newpage
\subsection{Развёрнутый график движения}

\begin{table}[h!]
	\centering
	\resizebox{0.95\textwidth}{!}{%
		\begin{tabular}{|>{\centering\arraybackslash}m{0.045\linewidth}
				|>{\centering\arraybackslash}m{0.02\linewidth}
				|>{\centering\arraybackslash}m{0.43\linewidth}
				|>{\centering\arraybackslash}m{0.09\linewidth}
				|>{\centering\arraybackslash}m{0.1\linewidth}
				|>{\centering\arraybackslash}m{0.05\linewidth}
				|>{\centering\arraybackslash}m{0.09\linewidth}
				|>{\centering\arraybackslash}m{0.13\linewidth}|}
			\hline						
			Дата	&	\begin{turn}{90}День\end{turn}	&	Участок маршрута	&	Км с $k=1.2$	&	Набор /сброс, м	&	ЧХВ	&	Высота ночёвки, м	&	Способы передвижения	\\
			\hline
			
			03.08	&	1	&	кур. Джилысу~--- д.р. Чон-Кызылсу~--- д.р. Саватор	&	6.0	&	$+550$\newline$-260$	& 5:07	&	2640	&	~Машина,\newline Пешком	\\
			\hline
			04.08	&	2	&	м.н.~--- м.н. в верховьях д.р. Саватор 	&	5.3	& $+900$\newline$-105$		& 5:15		& 3435		&	Пешком	\\
			\hline
			05.08	&	3	&	м.н.~--- \textbf{пер. Саватор (1А, 3860)}~--- м.н. (точка Обед)	&	5.2	& $+425$\newline$-280$		& 7:45	& 3580		&	Пешком	\\
			\hline
			06.08	&	4	&	м.н.~--- д.р. Кичи-Кызылсу	&	6.5	& $+245$\newline$-305$		& 4:24		& 3520		&	Пешком	\\
			\hline
			07.08	&	5	&	м.н.~--- \textbf{пер. Перемётный (1А, 3985)}~--- д.р. Джуукучак	&	12.4	& $+465$\newline$-1075$		& 10:14	& 2910		&	Пешком	\\
			\hline
			08.08	&	6	&	м.н.~--- \textbf{пер. Ашутор Западный (1А, 3620)}~--- д.р. Ашукашкасу~--- слияние р. Ашукашкасу и р. Джууку, заброска	&	13.8 	& $+710$\newline$-1100$		& 6:47		& 2520		&	Пешком	\\
			\hline
			09.08	&	7	&	м.н.~--- д.р. Джууку~--- д.р. Иттиши	&	9.9	& $+755$\newline$-0$		& 5:26		& 3275		&	Пешком	\\
			\hline
			10.08	&	8	&	м.н.~--- \textbf{пер. Иттиш (1А, 3890)}~--- оз. под перевалом, м.н.	&	9.5	& $+615$\newline$-25$		& 6:23		& 3865		&	Пешком	\\
			\hline
			11.08	&	9	&	м.н.~--- д.р. Ит-тиши~--- \textbf{безымянный пер. (н/к, 4035)}~--- оз. Кашкасу	&	12.9	& $+170$\newline$-140$		& 5:46		& 3895		&	Пешком	\\
			\hline
			12.08	&	10	&	м.н.~--- \textbf{рад. верш. Марс (н/к, 4334)}~--- оз. Кашкасу ~--- \textbf{пер. Кашкасу (1А, 3865)}~--- верховья д.р. Ашукашкасу	&	13.1	& $+439$\newline$-1024$		& 7:48		& 3310		&	Пешком	\\
			\hline
			13.08	&	11	&	м.н.~--- д.р. Ашукашкасу~--- слияние р. Ашукашкасу и р. Джууку~--- д.р. Джууку~--- слияние р. Джуукучак и Джууку (финиш) &	26.1	& $+0$\newline$-1310$		& 6:47		& -		&	Пешком 	\\
			\hline
			\multicolumn{3}{|c|}{\textbf{\textit{\Large{Итого:}}}} & \large{\textbf{120.7}} & \large{$\mathbf{+5274}$\newline$\mathbf{-5624}$	}	& \multicolumn{3}{c|}{\large{\textbf{71:42}\newline\textbf{2д 11ч 42мин}}} \\
			\hline
		\end{tabular}
}	
	
\end{table}

\clearpage

\subsection{Определение категории сложности маршрута}
Оценка категории сложности лыжного маршрута проводилась согласно методике категорирования лыжных туристских маршрутов \cite{ftsr_ski_2023}

\clearpage