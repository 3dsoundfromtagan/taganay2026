\section{Организация и проведение похода}
\subsection{Цели и задачи маршрута. Выбор нитки маршрута}
При разработке и планировании маршрута я, как руководитель, руководствовался следующими соображениями:
\begin{enumerate} 
	\item \textbf{Высотность, транспортная доступность горного района и стоимость трансфера.}
	Подавляющая часть группы (шесть человек из восьми) имела официальный или неофициальный опыт горных походов до 2 к.с по Кавказу \cite{Snegovskaya2024} или Алтаю, и вся группа имела высотный опыт ночёвки не менее 2900 м \cite{ostapiv2025}. Это позволило выбрать в качестве района новичкового похода не Гвандру или Архыз, как это обычно бывает, а новый для участников и руководителя горный район Центральной Азии~--- Тескей-Ала-Тоо. Высоты в этом районе, в среднем, на 500 м больше, чем на Кавказе, а его транспортную доступность можно назвать приемлемой для новичковых походов: 4 часа самолётом до Бишкека и еще 7~--- до места старта в горах. Стоимость трансфера выше, чем на Кавказе, но ниже, чем на Алтае. 
		
	\item \textbf{Концентрированность препятствий.}
	Дополнительным фактором в сторону выбора Тескей-Ала-Тоо послужило также и то, что в отличие, например, от Алтая, хребтовка района (основной хребет и серия параллельных отрогов), как и на Кавказе, позволяет осуществить подход под многие перевалы в течение одного дня, миновав долгие и занудные забеги по долинам, и, как следствие, поддерживать интерес группы на приемлемом уровне.
	
	\item \textbf{Разнообразие рельефа.} 
	С методической точки зрения, а также, опять-таки, для поддержания интереса группы, хотелось продемонстрировать участникам как можно больше разнообразных типов рельефа. Район позволяет в походах 1 к.с. продемонстрировать все виды рельефа, за исключением, пожалуй, снежного. <<Изюминкой>> района является возможность пересечения главного хребта через перевалы 1А (разительно отличающихся от классических <<единичек А>>) с осмотром высокогорных болот~--- сыртов.
	
	\item \textbf{Эффект кульминации.}
	У меня как у руководителя было глубокое убеждение, что первый поход должен обладать понятным, с позволения сказать, сюжетом и иметь свою кульминацию. В нашем случае таким сюжетом было движение через отроги главного хребта с преодолением разнообразных перевалов (последовательно: травянисто-осыпной, ледовый, травянистый), а затем~--- кольцо через главный хребет с <<неклассическими>> перевалами на южную сторону, восхождением на обзорную точку~--- вершину Марс. Конец маршрута~--- забег по среднегорью с финишем на локальной достопримечательности <<Красные скалы>>~--- также должен был добавить в поход разнообразия и красоты. Наконец, по моему глубокому убеждению, финальным аккордом походов по Кыргызстану должен быть хотя бы  однодневный (а лучше двухдневный) отдых на Иссык-Куле.
	
	\item\textbf{Помощь в планировании}
	Фактор, который формально не был в списке определяющих критериев, но по факту являлся таковым~--- это искренняя заинтересованность и помощь в планировании и организации маршрута от человека, через которого организовывалась логистика похода~--- Юрия Траченко. За эту помощь выражаю ему огромную благодарность.
	
\end{enumerate} 

\subsection{Логистика и газ}
ЛОГИСТИКИ ХОЧЕЦА



Газ также заказывали через Юрия Траченко. Стоимость около 1050~\faRub~ за баллон, заказывать нужно было в конце апреля, однако была возможность бесплатно отказаться от 30\% газа за определенное время до начала похода. Раз брали из расчёта 45 гр/чел/день, то есть 5 баллонов на первое кольцо и 6~--- на второе. Фактически, к концу первого кольца у нас осталось около трети одного баллона, а по случаю схода с маршрута трех участников мы отдали им один баллон. К концу похода у нас остались один неначатый баллон и один полупустой, то есть, в целом, расчёт количества газа можно считать правильным (с небольшим запасом на непредвиденный случай). Снег не топили, всегда использовали ветрозащитные шторки и базальтовую ткань.



\subsection{Аварийные выходы из маршрута и его запасные варианты}
\textbf{Аварийными выходами} с маршрута являлись:
\begin{itemize}
	\item На первом этапе: спуск к т/б <<Глобус>>;
	\item На втором этапе: спуск к а/л <<Узункол>>;
	\item На третьем этапе: спуск к погранзаставе <<Актюбе>> (Хурзук).
\end{itemize}
\textbf{Запасными вариантами} маршрута являлись:
\begin{itemize}
	\item Замена пер. Уллу-Кёль Восточный (1А$^\star$, 3050) на пер. \textbf{Уллу-Кёль Нижний (н/к, 2933)};
	\item Отказ от пер. Перемётный (1А, 3255), спуск по д.р. Чунгур-Джар;
	\item Отказ от пер. Хотютау (1А$^\star$), спуск по д.р. Кубань к погранзаставе <<Хурзук>>
\end{itemize}
\subsection{Изменение маршрута и их причины}
Маршрут пройден без изменений.

\subsection{Обеспечение безопасности на маршруте}
Группа была зарегистрирована в отделении МЧС Кыргызстана по Иссык-Кульской области (телефон +996555004214, связь по WhatsApp). За 10 дней до похода были переданы сведения о составе группы, сроках и маршруте, в ответ был получен регистрационный номер и просьба проинформировать дежурного о начале и завершении похода.

\alert{Дима, это с тебя Для регулярного обмена сообщениями, отслеживания положения группы на карте, а также возможности экстренной связи, в группе имелся спутниковый треккер IRIDIUM Rockstar 360. Стоимость аренды треккера в <<Альпиндустрии>> на 15--21 день составила 7100~\faRub, залог~--- 50000~\faRub. Нам повезло попасть на демострационный период тарифа треккера, в связи с чем все сообщения были для нас безлимитны и бесплатны. Предварительное тестирование треккера в Москве показало, что спутниковые сигналы в столице эффективно глушатся: сообщения приходили не чаще раза в сутки. В походе с приёмом и отправкой сообщений и координат на сервер проблем не возникало, среднее время отправки составляло 30 минут.}

Каждый участник самостоятельно оформлял на себя индивилуальный страховой полис. Выбрали страховую фирму <<Согласие>>, ассист Balt Assistance, опция <<Треккинг свыше 1500 м>>, размер страховой защиты 35000~USD,  вид отдыха <<Спорт Экстрим>>. Стоимость полиса составила 8234~\faRuble~с человека.

\subsection{Перечень наиболее интересных природных и исторических объектов, занятий на маршруте}
\begin{enumerate}[noitemsep,topsep=0pt,parsep=0pt,partopsep=0pt]
	\item Широкие красивые долины северной стороны хребта: Чон-Кызыл-Суу, Киче-Кызыл-Суу, Ашукашкасу, Джукучак; 
	\item Сырты на южной стороне хребта; 
	\item Цепочка озёр Кашкасу с чистой водой; 
	\item Вершина Марс, с которой открываются виды на Акширак, Кумтор, сырты и, в хорошую погода, на семитысячники; 
	\item Перевал Кашкасу, через который в былые годы перегоняли лошадей. Следы этого остаются в ущелье до сих пор, жутковато и атмосферно; 
	\item <<Красные скалы>> на слиянии р. Джууку и Джукучак;
	\item Иссык-Куль.
\end{enumerate}

\paragraph{Темы практических занятий:}

\begin{itemize}
	\item Техника передвижения по травянистым и осыпным склонам;
	\item Техника несложных бродов поодиночке;
	\item Техника передвижения по льду.
\end{itemize}

\newpage
\subsection{Развёрнутый график движения}
\alert{Дима, это на тебе, пожалуйста}
\begin{table}[h!]
	\centering
	\resizebox{0.95\textwidth}{!}{%
		\begin{tabular}{|>{\centering\arraybackslash}m{0.045\linewidth}
				|>{\centering\arraybackslash}m{0.02\linewidth}
				|>{\centering\arraybackslash}m{0.43\linewidth}
				|>{\centering\arraybackslash}m{0.09\linewidth}
				|>{\centering\arraybackslash}m{0.1\linewidth}
				|>{\centering\arraybackslash}m{0.05\linewidth}
				|>{\centering\arraybackslash}m{0.09\linewidth}
				|>{\centering\arraybackslash}m{0.13\linewidth}|}
			\hline						
			Дата	&	\begin{turn}{90}День\end{turn}	&	Участок маршрута	&	Км с $k=1.2$	&	Набор /сброс, м	&	ЧХВ	&	Высота ночёвки, м	&	Способы передвижения	\\
			\hline
			
			03.08	&	1	&	кур. Джилысу~--- д.р. Чон-Кызылсу~--- д.р. Саватор	&	11.3	&	$+550$\newline$-260$	& 5:07	&	2640	&	~Машина,\newline Пешком	\\
			\hline
			04.08	&	2	&	м.н.~--- м.н. в верховьях д.р. Саватор 	&	6.0	& $+900$\newline$-105$		& 5:15		& 3435		&	Пешком	\\
			\hline
			05.08	&	3	&	м.н.~--- \textbf{пер. Саватор (1А, 3860)}~--- м.н. (точка Обед)	&	5.8	& $+425$\newline$-280$		& 7:45	& 3580		&	Пешком	\\
			\hline
			06.08	&	4	&	м.н.~--- д.р. Кичи-Кызылсу	&	9.1	& $+245$\newline$-305$		& 4:24		& 3520		&	Пешком	\\
			\hline
			07.08	&	5	&	м.н.~--- \textbf{пер. Перемётный (1А, 3985)}~--- д.р. Джуукучак	&	13.8	& $+465$\newline$-1075$		& 10:14	& 2910		&	Пешком	\\
			\hline
			08.08	&	6	&	м.н.~--- \textbf{пер. Ашутор Западный (1А, 3620)}~--- д.р. Ашукашкасу~--- слияние р. Ашукашкасу и р. Джууку, заброска	&	16.7 	& $+710$\newline$-1100$		& 6:47		& 2520		&	Пешком	\\
			\hline
			09.08	&	7	&	м.н.~--- д.р. Джууку~--- д.р. Иттиши	&	11.2	& $+755$\newline$-0$		& 5:26		& 3275		&	Пешком	\\
			\hline
			10.08	&	8	&	м.н.~--- \textbf{пер. Иттиш (1А, 3890)}~--- оз. под перевалом, м.н.	&	12.0	& $+615$\newline$-25$		& 6:23		& 3865		&	Пешком	\\
			\hline
			11.08	&	9	&	м.н.~--- д.р. Ит-тиши~--- \textbf{безымянный пер. (н/к, 4035)}~--- оз. Кашкасу	&	16.1	& $+170$\newline$-140$		& 5:46		& 3895		&	Пешком	\\
			\hline
			12.08	&	10	&	м.н.~--- \textbf{рад. верш. Марс (н/к, 4334)}~--- оз. Кашкасу ~--- \textbf{пер. Кашкасу (1А, 3865)}~--- верховья д.р. Ашукашкасу	&	15.3	& $+439$\newline$-1024$		& 7:48		& 3310		&	Пешком	\\
			\hline
			13.08	&	11	&	м.н.~--- д.р. Ашукашкасу~--- слияние р. Ашукашкасу и р. Джууку~--- д.р. Джууку~--- слияние р. Джуукучак и Джууку (финиш) &	26.4	& $+0$\newline$-1310$		& 6:47		& -		&	Пешком 	\\
			\hline
			\multicolumn{3}{|c|}{\textbf{\textit{\Large{Итого:}}}} & \large{\textbf{143.7}} & \large{$\mathbf{+5274}$\newline$\mathbf{-5624}$	}	& \multicolumn{3}{c|}{\large{\textbf{71:42}\newline\textbf{2д 11ч 42мин}}} \\
			\hline
		\end{tabular}
}	
	
\end{table}



\clearpage