\subsection{12 августа. Пер. Кашкасу (1A)}
\textit{Метеоусловия: Весь день переменная облачность; к 16.50 распогодилось, ясно.}

\begin{figure}[h!]
	\centering
	\includegraphics[angle=0, width=0.5\linewidth]{pics/maps/12}
	\label{fig:12}
\end{figure}

Подъём дежурных в 4:30, в 05:40 группа (кроме Маши) выходит в бодром темпе на в. Марс. Подъём на вершину прост. От лагеря подходим по долине под склон (1 км, около 20 мин ЧХВ), далее поднимаемся по широкому кулуару~--- руслу ручья (можно набрать воды) в амфитеатр, что занимает ещё около 45 мин ЧХВ. Далее поднимаемся на сам амфитеатр и идём по предвершинному плато к вершине (30 мин ЧХВ). В 07:18 выходим на вершину и устраиваем получасовой привал. Погода великолепная, виды потрясающие.

\begin{figure}[h!]
	\centering
	\includegraphics[width=0.7\linewidth]{pics/12/mars_raise.jpg}
	\caption{Слева: подъём по кулуару, справа: движение по предвершинному плато}
	\label{fig:mars_raise.jpg}
\end{figure}

\begin{figure}[h!]
	\centering
	\includegraphics[width=0.7\linewidth]{pics/12/IMG_3590.jpg}
	\caption{Группа на в. Марс (н/к, 4350)}
	\label{fig:IMG_3590.jpg}
\end{figure}


\begin{figure}[h!]
	\centering
	\includegraphics[width=0.9\linewidth]{pics/12/IMG_3584.jpg}
	\caption{Вид на хр.~Акшийрак с вершины}
	\label{fig:IMG_3584.jpg}
\end{figure}

К 9:15 возвращаемся, обедаем, и в 10:58 начинаем движение вдоль левого пхд берега системы озёр. 15 мин, с 11:15 по 11:30, ушло на брод ручья, стекающего с ледника Кашкасу (№380). После этого ручья река, текущая по дну долины, меняет своё направление и следует вместе с нами в сторону перевала, на север.

\begin{figure}[h!]
	\centering
	\includegraphics[width=0.7\linewidth]{pics/12/20250812_110552.jpg}
	\caption{Путь на пер. Кашкасу}
	\label{fig:toKashkasu}
\end{figure}

Дальнейшее движение до седловины несложно с технической точки зрения, но необходимость огибания озёр не раз создёт ложное ощущение близости седловины. Места атмосферные: слева и справа на много сотен метров высится главный хребет, река обманывает гравитацию, а после скрывается в камнях, разбросаны лошадиные и человеческие кости.

\begin{figure}[h!]
	\centering
	\includegraphics[width=0.35\linewidth]{pics/12/20250812_122606.jpg}
	\caption{Атмосферно}
	\label{fig:20250812_122225.jpg}
\end{figure}


Отличный маркер седловины~--- два больших камня, похожие на ворота. Выходим к туру в 13:35. Однако перед ними тропа плотно прижимается к реке, а затем река уходит мод камни: нужно быть аккуратным при движении на этом участке.

\begin{figure}[h!]
	\centering
	\includegraphics[width=0.5\linewidth]{pics/12/IMG_3620.JPG}
	\caption{Движение к перевалу. По горизонтали до седловины менее 500 м. Вода активно уходит под морену, с каждым шагом её всё меньше}
	\label{fig:IMG_3620.JPG}
\end{figure}

\begin{figure}[h!]
	\centering
	\includegraphics[width=0.7\linewidth]{pics/12/IMG_3624.JPG}
	\caption{пер. Кашкасу, вид в д.р. Ашукашкасуу}
	\label{fig:IMG_3624.JPG}
\end{figure} 

\begin{figure}[h!]
	\centering
	\includegraphics[width=0.7\linewidth]{pics/12/IMG_3626.JPG}
	\caption{пер. Кашкасу, вид в д.р. Кашкасу (озёра)}
	\label{fig:IMG_3626.JPG}
\end{figure}

Виды потрясающие: с ледопадов на бортах долины с грохотом летят камни, в левом пхд моренном кармане зияют ледовые пещеры. 

Начинаем спуск в 14:00. Он проходит по гребням моренных валов. 80 вертикальных метров спустя доходим до небольшого озера. При непреодолимом желении здесь можно встать на ночёвку. Ещё 100 вертикальных метров спустя путь спуска раздваивается. Тропа на OSM, треки Ковинова \cite{kovinov2021} и Родиной \cite{rodina2012} спускаются в правый моренный карман, а трек Сергеева \cite{sergeev2024} ведёт по моренным гребням, там же виднеются турики. Выбираем второй вариант как более удобный и безопасный.


\begin{figure}[h!]
	\centering
	\includegraphics[width=0.7\linewidth]{pics/12/20250812_144637}
	\caption{Спуск с пер. Кашкасу}
	\label{fig:20250812_144637}
\end{figure}

Спускаться технически и физически нетрудно, но достаточно занудно. Везде идём ногами, даже за камни придерживаться нигде не приходится. За несколько переходов добираемся до языка моренного вала, за которым начинаются зелёные площадки и из-под морены показывается р. Ашукашкасуу. Группа Кати уже здесь и вовсю обедает. С языка мы спустились по левому пхд его борту, однако, возможно, если свернуть в правый моренный карман, спуск будет немного положе.

\begin{figure}[h!]
	\centering
	\includegraphics[width=0.7\linewidth]{pics/12/IMG_3641.JPG}
	\caption{Вид с языка моренного вала на м.н}
	\label{fig:IMG_3641.JPG}
\end{figure}

Катина группа вскоре отправляется дальше по долине, а мы встаём на ночёвку в 16:50 на травянистых площадках (рис.~\ref{fig:IMG_3641.JPG}). Вода, которая стекает с ледника (на фото два правых ручья) мутная из-за активного дневного таяния последнего, воду берём из притока, вытекающего из-под <<нашей>> морены. Впрочем, утром вода была чистой во всех ручьях.

Координаты м.н.: N 42.05625° E 78.05825°.

ЧХВ: 7:48, ОХВ: 11:05.


\clearpage
