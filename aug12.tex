\subsection{12 августа. Пер. Кашкасу (1A)}
\textit{Метеоусловия: Весь день переменная облачность; к 16.50 распогодилось, ясно.}

\begin{figure}[h!]
	\centering
	\includegraphics[angle=0, width=0.5\linewidth]{pics/maps/12}
	\label{fig:12}
\end{figure}

Подъём дежурных в 4:30, в 05:40 группа (кроме Маши) выходит в бодром темпе на в. Марс. Подъём на вершину прост. От лагеря подходим по долине под склон (1 км, около 20 мин ЧХВ), далее поднимаемся по широкому кулуару~--- руслу ручья (можно набрать воды) в амфитеатр, что занимает ещё около 45 мин ЧХВ. Далее поднимаемся на сам амфитеатр и идём по предвершинному плато к вершине (30 мин ЧХВ). В 07:18 выходим на вершину и устраиваем получасовой привал. Погода великолепная, виды потрясающие.


\begin{figure}[h!]
	\centering
	\includegraphics[width=0.7\linewidth]{pics/12/mars_raise.jpg}
	\caption{Слева: подъём по кулуару, справа: движение по предвершинному плато}
	\label{fig:mars_raise.jpg}
\end{figure}

\begin{figure}[h!]
	\centering
	\includegraphics[width=0.7\linewidth]{pics/12/IMG_3590.jpg}
	\caption{Группа на в. Марс (н/к, 4350)}
	\label{fig:IMG_3590.jpg}
\end{figure}


К 9:15 возвращаемся, обедаем и в 10:58 начинаем движение вдоль левого пхд берега системы озёр. 15 мин, с 11:15 по 11:30, ушло на брод ручья, стекающего с ледника Кашкасу (№380). После этого ручья река, текущая по дну долины, меняет своё направление и следует вместе с нами в сторону перевала, на север.

Дальнейшее движение до седловины несложно с технической точки зрения, но необходимость огибания озёр не раз создёт ложное ощущение близости седловины. Места атмосферные: слева и справа на много сотен метров высится главный хребет, река обманывает гравитацию, а после скрывается в камнях, разбросаны лошадиные и человеческие кости.

Отличный маркер седловины~--- два больших камня, похожие на ворота. Выходим к туру в 13:35. ФОТО. Виды потрясающие: с ледопадов на бортах долины с грохотом летят камни, в левом пхд моренном кармане зияют ледовые пещеры. Мы же, посмотрев на всё это великолепие, от греха подальше решаем двигаться по моренным гребням, благо, то и дело встречаются турики, и такое движение не представляет технической сложности, хоть и несколько утомительно. К 16:50 спускаемся с моренных валов на место выхода р. Ашукашкасу из-под них и встаём на ночёвку на речном разливе.

Координаты м.н.: N 42.05625° E 78.05825°.

ЧХВ: 7:48, ОХВ: 11:05.
\begin{figure}[h!]
	\centering
		\includegraphics[width=0.7\linewidth]{pics/10/camp_10}
	\caption{Место ночёвки 10-11.08 — \textcolor{red}{заменить!!!!!}}
	\label{fig:camp_10}
\end{figure}

\clearpage
