\section{Отчёт медика}



	
	\begin{itemize}
		\item \textbf{Хронология похода}
		
		\begin{enumerate}
			\item \textbf{3 августа} -- Все участники похода здоровы. Без проблем прошли вверх по долине, набор высоты небольшой, тропа широкая.
			
			\item \textbf{4 августа} -- Тропа пошла круто вверх, подъем стал даваться труднее. Ночевка на высоте больше 2500 метров у нескольких членов команды вызвала учащенное сердцебиение ночью и прерывистый сон.
			
			\item \textbf{5 августа} -- Обнаружились первые признаки недомогания у двух членов группы -- температура чуть выше 37, слабость. Предположили, что это реакция на большую высоту. Первый перевал прошли очень медленно.
			
			\item \textbf{6 августа} -- У заболевших продолжила подниматься температура до 37,5. Всем соседям по палатке был выдан арбидол в целях профилактики. Один из заболевших начал принимать противовирусные препараты.
			
			\item \textbf{7 августа} -- Температура у двух первых заболевших поднялась до 38, появились и другие члены группы с температурой 37,2. Перед прохождением перевала сбили температуру 38 с помощью ибупрофена и терафлю. У одного из заболевших позже появляется дискомфорт в ЖКТ, он принимает смекту.
			
			\item \textbf{8 августа} -- Температура у первых заболевших снижается, хотя слабость и недомогание сохраняются. Заболевшие позже чувствуют себя хуже, с трудом проходят перевал. От дискомфорта в ЖКТ принимается энтерол, но заметного эффекта это не дает, весь день человек почти ничего не ест. Спустились в долину к заброске. Есть возможность сойти с маршрута в этой точке, т.к. в долине проходит автомобильная дорога. Трое заболевших принимают решение сойти с маршрута, т.к. их самочувствие не улучшается, а второе кольцо сложнее первого и не имеет точек досрочного завершения маршрута.
			
			\item \textbf{9 августа} -- Продолжаем маршрут. Состояние членов команды удовлетворительное, температуры выше 37 ни у кого нет. Есть проблемы с ногами -- из пяти человек четверо обрабатывают мозоли бетадином.
			
			\item \textbf{10 августа} -- Тяжелый подъем на перевал по курумнику. Ночевка на 3865 м. Не проходит сильный насморк с заложенными ушами у одного из участников. Пытаемся облегчить состояние каплями в нос.
			
			\item \textbf{11 августа} -- Переход на высотах около 4000 м, состояние участников стабильно.
			
			\item \textbf{12 августа} -- Четверо из пяти человек совершили радиальное восхождение на вершину Марс, затем вся группа осуществила несколько бродов реки, частично промокнув, прошла перевал и спустилась по моренам до места ночевки на 3310 м. Все очень устали. У нескольких человек кровоточит слизистая носа (возможно, имеет смысл брать какое-то ранозаживляющее масло или крем на этот случай и обрабатывать нос изнутри) из-за холодного и сухого воздуха на больших высотах.
			
			\item \textbf{13 августа} -- Спуск по долине к месту финиша. Было принято решение пройти все расстояние за один день, чтобы побыстрее спуститься с гор и избежать еще одной ночевки под надвигающимся дождем в палатке.
		\end{enumerate}
		
		\item \textbf{Выводы}
		
		\begin{itemize}
			\item В целом, привезенная из Москвы инфекция сильно ослабила почти всех участников группы и усложнила прохождение маршрута.
			\item Возможно, в личной аптечке каждого должен быть курс противовирусного препарата, который он использует в обычной жизни.
		\end{itemize}
		
	\end{itemize}
	


	
	\begin{itemize}
		\item \textbf{Аптечка на группу}
		
		\begin{enumerate}
			\item \textbf{Обезболивающие} (суточную дозу ни одного из них нельзя превышать, так что в случае необходимости нужно комбинировать несколько препаратов):
			\begin{itemize}
				\item Кетанов -- сильное обезболивающее при травмах (можно в таблетках, но колоть местно действует быстрее);
				\item мазь «Долобене» -- снимает боль в мышцах и суставах от растяжений и ударов;
				\item Пенталгин -- парацетамол+кофеин, от головной боли и высокой температуры;
				\item Нурофен -- ибупрофен, от боли и высокой температуры;
				\item Аспирин -- ацетилсалициловая кислота, сбивает температуру и разжижает кровь (хорошо, как дополнительное жаропонижающее при перегреве);
				\item Но-шпа -- снимает спазмы гладкой мускулатуры в животе и в голове.
			\end{itemize}
			
			\item \textbf{От аллергии} (в том числе, при укусах насекомых и змей):
			\begin{itemize}
				\item Супрастин -- сильный, но дает сонливость;
				\item Зодак;
				\item Дексаметазон -- только экстренно в случае укусов ядовитыми животными или не сбиваемой температуры (можно в таблетках, но инъекции действуют быстрее и сильнее);
			\end{itemize}
			
			\item \textbf{Отравления} (тошнота, рвота, понос, в том числе, бактериальной и вирусной природы):
			\begin{itemize}
				\item Смекта -- натуральный адсорбент из глины, перед ней лучше промыть желудок водой (или бледно-розовым раствором марганцовки);
				\item Энтерол -- адсорбент в том числе и вирусов, после смекты, если нет высокой температуры;
				\item Нифуроксазид -- после смекты, если поднялась температура;
				\item Фурозолидон -- после нифуроксазида через сутки, если не помог;
				\item Регидрон -- после тошноты и поноса, чтобы вернуть соли в организм;
				\item Бифидумбактерин -- заселение симбиотических бактерий в кишечный тракт после приема антибактериальных препаратов;
				\item Нольпаза -- от изжоги;
			\end{itemize}
			
			\item \textbf{Антибактериальные препараты} (долго держится высокая температура, открытые раны):
			\begin{itemize}
				\item Панцеф -- антибиотик внутрь, если несколько суток не снижается температура;
				\item Стрептоцид -- наружный антибиотик -- полоскать горло при ангине, присыпать раны;
				\item Банеоцин (бацидерм) -- антибактериальный порошок на мокнущие раны;
				\item Фурацилин -- растворить одну таблетку в 100 мл воды и промывать или накладывать пропитанную раствором салфетку на гнойные раны;
				\item Тобрекс -- капли в глаза при воспалении;
			\end{itemize}
			
			\item \textbf{Ранозаживляющие}:
			\begin{itemize}
				\item Пантенол спрей -- распылить на место ожога, дать подсохнуть;
				\item мазь «Спасатель»;
			\end{itemize}
			
			\item \textbf{Противовирусные}:
			\begin{itemize}
				\item Арбидол -- внутрь для профилактики, если кто-то привез ОРВИ;
				\item Ацикловир -- мазь снаружи от герпеса («простуды»).
			\end{itemize}
			
			\item \textbf{Облегчающие течение ОРВИ}:
			\begin{itemize}
				\item Терафлю -- развести порошок и выпить, если нет возможности отлежаться, надо идти;
				\item Санорин -- капли в нос облегчают течение насморка;
				\item «Звездочка» -- растереть виски, между бровей.
			\end{itemize}
			
			\item \textbf{От высокого давления}:
			\begin{itemize}
				\item Кантоприл -- начинать с четверти таблетки, снижает давление, по возможности, контролировать давление аппаратом, можно дать при инфаркте.
			\end{itemize}
			
			\item \textbf{Перевязочные средства при травмах и растяжениях}:
			\begin{itemize}
				\item Бинты широкие;
				\item Бинты узкие;
				\item Стерильные марлевые салфетки -- на раны после обработки;
				\item Эластичный бинт -- на растяжения связок для фиксации;
				\item Пластырь бактерицидный -- большие пластины на раны и мозоли;
				\item Ножницы маникюрные;
				\item Пластырь рулонный для фиксации повязок;
				\item Бетадин -- слабый раствор йода на неглубокие раны, в том числе, мозоли (лучше в виде мази);
				\item Хлоргексидин -- первичная промывка неглубокой раны;
				\item Перекись водорода -- первичная промывка глубоких, узких ран.
			\end{itemize}
			
		\end{enumerate}
		
		\item \textbf{Личная аптечка участника}:
		\begin{itemize}
			\item Противовирусные таблетки;
			\item Капли в нос и другие любимые средства, облегчающие течение ОРВИ;
			\item Любимое обезболивающее и жаропонижающее;
			\item 1-2 пакетика смекты;
			\item Большой запас пластырей от мозолей разных размеров;
			\item 1-2 небольших бинта;
			\item эластичный бинт или другой фиксатор;
			\item индивидуальные специфические лекарства.
		\end{itemize}
		
	\end{itemize}
	
\clearpage
