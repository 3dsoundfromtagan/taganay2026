\subsection{06 августа. д.р. Киче-Кызыл-Суу}
\textit{Метеоусловия: утром ясно, днём слабый дождь, вечером облачно}

\begin{figure}[h!]
	\centering
	\includegraphics[angle=0, width=0.5\linewidth]{pics/maps/06}
	\label{fig:06}
\end{figure}

Встаём в 06:00, отбиваем наледь с палатки. Выходим в 08:30 и движемся вниз по склону в д.р. Киче-Кызыл-Суу. За нашими спинами остаётся спуск с пер. Каратакия (1А), который днём ранее проходила Катина группа, перед нами~--- вид на склон пер. Кашкатор Северный, по которому Катина группа должна карабкаться сейчас. Мы их не замечаем (а они нас~--- да \smiley). Слева пхд, в верховьях Киче-Кызыл-Суу, виднеется пер. Перемётный. Пока спускаемся в долину, размышляем, на какой перевал идти (выбор Кашкатора, потенциально, способствовал бы воссоединению с Катиной группы, а Перемётный объективно красивее и интереснее, благо оба перевала перед нами как на ладони). Наш спуск приятный, уклон не превышает 20\degree.

Не торопясь спускаемся на дно долины в 10:45. Отдыхаем, спустя некоторое время устраиваем обед. Окончательно решаем идти на пер. Перемётный и использовать запасной день.

Выходим с места обеда в 13:30, решаем вскарабкаться на борт каньона и встать на ночёвку у подножия моренных валов. Поднимаемся по правому берегу реки, предпочитая травянистые участки каменистым. За 1.5 часа ЧХВ, в 15:45, добираемся до м.н. Решаем не переходить на другую сторону реки, как в \cite{rodina2012}, только ради хороших площадок, так как и на правом берегу места вполне хорошие, да и двигаться на следующий день нужно здесь же. М.н. представляет из себя большое плоское поле, сложность для постановки палаток представляют только земляные бугорки и навоз. Остаток дня отдыхаем, кушаем и кружим на дроне над моренными валами, стараясь наметить наиболее оптимальный путь подъёма на перевал.

Координаты м.н.: N 42.12278° E 78.13331°.

ЧХВ:03:15, ОХВ: 9:43.


\clearpage